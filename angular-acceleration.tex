\input ../paper.tex

\papertitle{The Angular Acceleration Vector}

\paperheading{Discovery}

Our goal is to find the time derivative of the angular velocity vector
$\sv\omega$ in terms of the torque on a body. We will frequently be crossing
vectors with $\sv\omega$, so let us establish some matrix notation for this
operation:
$$
\sv\omega \x \sv r =
\left[\matrix{
0 & -\omega_z & \omega_y \cr
\omega_z & 0 & -\omega_x \cr
-\omega_y & \omega_x & 0 \cr
}\right]
\left[\matrix{
x \cr y \cr z
}\right]
=
\Omega\?r
$$

Now posit some frame which rotates according to $\sv\omega$. There will be a
unitary matrix $R^T$ which transforms coordinates from inertial space into
rotating space. Taking a cross product in rotating space must yield the same
result as taking it in inertial space and then transforming the resulting
coordinates:
$$
\Omega' r' = \R R^T\Omega\? r
$$
Since $r' = \R R^Tr$ we can conclude that
$$
\Omega = \R R\Omega'\R R^T
\hskip15pt
\Omega' = \R R^T\Omega\R R
$$
These relations allow us to take a cross product with $\sv\omega$ in one frame
given the coordinates of $\sv\omega$ in the other.

We need to establish two more lemmas before we can tackle the problem at hand.
Recall that the basis vectors of the rotating frame are the columns of the
matrix $R$. The time derivatives of these vectors can be found by crossing
them with $\sv\omega$, and so we have
$$
\dot \R R = \Omega \R R
$$
Finally, consider the following
$$
\R L = \R I\omega
\hskip12pt
\R L' = \R I'\omega'
\hskip12pt
\R L' = \R R^T\R L
\hskip12pt
\omega' = \R R^T\omega
$$
We can combine these to get
$$
\R I = \R R\R I'\R R^T
$$
Now we can proceed very rapidly:
$$
\tau
=
\left.\q{d}{dt}\middle(\R I\omega\right)
=
\left.\q{d}{dt}\middle(\R R\R I'\R R^T\R R\,\omega'\right)
=
\left.\q{d}{dt}\middle(\R R\R I'\omega'\right)
=
\dot \R R\R I'\omega' + \R R\R I'\dot\omega'
=
\Omega\R R\R I'\omega' + \R R\R I'\dot\omega'
$$
Next we multiply on the left by $\R R^T$:
$$
\R R^T\tau = \R R^T\Omega\R R\R I'\omega' + \R I'\dot\omega'
$$
Finally we apply $\tau' = \R R^T\tau$ and $\Omega' = \R R^T\Omega \R R$ to get
$$
\tau' = \Omega'\R I'\omega' + \R I'\dot\omega'
$$
When we expand this out in component form we get Euler's equations:
$$
\eqalign{
\tau_x 
&=
\omega_y\omega_z\(\lambda_3 - \lambda_2\)
+ \lambda_1\dot\omega_x
\cr
\noalign{\vskip6pt}
\tau_y 
&=
\omega_z\omega_x\(\lambda_1 - \lambda_3\)
+ \lambda_2\dot\omega_y
\cr
\noalign{\vskip6pt}
\tau_z
&=
\omega_x\omega_y\(\lambda_2 - \lambda_1\)
+ \lambda_3\dot\omega_z
\cr
}
$$
Note that we have omitted writing the primes for brevity. Reference books
typically omit the primes as well, and they assume that the reader is aware
that the equations must be evaluated in the rotating frame. We can rearrange
these to find the coordinates of the angular acceleration vector in the
rotating frame
$$
\dot\omega_x
= \q{\tau_x - \omega_y\omega_z(\lambda_3 - \lambda_2)}{\lambda_1}
\hskip15pt
\dot\omega_y
= \q{\tau_y - \omega_z\omega_x(\lambda_1 - \lambda_3)}{\lambda_2}
\hskip15pt
\dot\omega_z
= \q{\tau_z - \omega_x\omega_y(\lambda_2 - \lambda_1)}{\lambda_3}
$$

We will often want to translate the angular acceleration back into inertial
coordinates. Translating a vector derivative out of a rotating frame typically
takes great care, but in this case the process simplifies nicely. Observe that
$$
\R R^T\omega = \omega' 
\hskip12pt\to\hskip12pt
\omega = \R R\omega'
\hskip12pt\to\hskip12pt
\dot\omega = \dot \R R\omega' + \R R\dot\omega'
\hskip12pt\to\hskip12pt
\dot\omega = \Omega\R R\omega' + \R R\dot\omega'
$$
This then reduces to
$$
\dot\omega = \Omega\R\omega + \R R\dot\omega'
$$
But $\Omega\R\omega = 0$, and so we have
$$
\dot\omega = \R R\dot\omega'
$$
In summary, if we want to find the angular acceleration vector in inertial
coordinates then we have to transform the torque and angular velocity into
rotating coordinates, apply Euler's equations, and then translate the result
back into inertial coordinates.

\bye

\paperheading{Discovery}

Recall that the derivative of a body's rotational angular momentum is given by
the rotational torques felt by the body:
$$
\sum_i \sv \rho_i \x \sv F_i = \q{d}{dt}\biggl[{\rm I}\sv\omega\biggr]
$$
where $\sv\rho_i$ is a vector that points from the body's center of mass to
one of its constituent particles, and $\sv F_i$ is the force felt by the given
particle. We will usually write this more compactly as
$$
\sv\tau = \q{d}{dt}\biggl[{\rm I}\sv\omega\biggr]
$$
We wish to find the vector $\dot{\sv\omega}$. Consider that
$$
\sv\tau
=
\q{d}{dt}\biggl[{\rm I}\sv\omega\biggr]
=
\[\q{d}{dt}{\rm I}\]\sv\omega
 + \rm I\?\dot{\sv\omega}
$$
The only problematic term here is the time derivative $d\?{\rm I}/dt$. Now, we
know that the tensor of inertia is
$$
{\rm I} = {\rm R^T}{\rm I}'{\rm R}
$$
where ${\rm I}'$ is a version of the tensor as seen from a frame fixed to the
body, and $\rm R$ is a matrix used to map from inertial coordinates to body
coordinates. By definition $\rm R$ is made up of three rows, each of which
hold the inertial coordinates of the body's moving basis vectors
(${\ihat}\?'$, ${\jhat}\?'$, and ${\khat}'$). The tensor ${\rm I}'$ is a
constant, and so we have
$$
\q{d}{dt}{\rm I}
=
\[\q{d}{dt}{\rm R^T}\]{\rm I}'{\rm R}
+ {\rm R^T}{\rm I}'\[\q{d}{dt}{\rm R}\]
$$
When we plug this into our original equation we get
$$
\eqalign{
\sv\tau
&=
\(
  \[\q{d}{dt}{\rm R^T}\]{\rm I}'{\rm R} +
  {\rm R^T}{\rm I}'\[\q{d}{dt}{\rm R}\]
\) \sv\omega
+ {\rm I}\?\dot{\sv\omega}
\cr
&=
\[\q{d}{dt}{\rm R^T}\]{\rm I}'{\rm R}\?\sv\omega
+ {\rm R^T}{\rm I}'\[\q{d}{dt}{\rm R}\]\sv\omega
+ {\rm I}\?\dot{\sv\omega}
\cr
&=
\[\q{d}{dt}{\rm R^T}\]{\rm I}'\sv\omega'
+ {\rm R^T}\?{\rm I}'\[\q{d}{dt}{\rm R}\]\sv\omega
+ {\rm I}\?\dot{\sv\omega}
\cr
&=
\[\q{d}{dt}{\rm R^T}\]\sv L'
+ {\rm R^T}{\rm I}'\[\q{d}{dt}{\rm R}\]\sv\omega
+ {\rm I}\?\dot{\sv\omega}
}
$$
Now we have to evaluate the derivatives of $\rm R$ and ${\rm R^T}$. We know
that
$$
\q{d\ihat'}{dt} = \sv\omega \x \ihat \hskip12pt
\q{d\jhat'}{dt} = \sv\omega \x \ihat \hskip12pt
\q{d\khat'}{dt} = \sv\omega \x \ihat
$$
But recall that $\rm R$ is just
$$
{\rm R} =
\left[
\matrix{
\ihat' * \ihat & \ihat' * \jhat & \ihat' * \khat \cr
\noalign{\vskip4pt}
\jhat\?' * \ihat & \jhat\?' * \jhat & \jhat\?' * \khat \cr
\noalign{\vskip4pt}
\khat' * \ihat & \khat' * \jhat & \khat' * \khat \cr
}\right]
$$
The time derivative is therefore
$$
\q{d}{dt}{\rm R} =
\left[
\matrix{
(\sv\omega \x \ihat') * \ihat &
(\sv\omega \x \ihat') * \jhat &
(\sv\omega \x \ihat') * \khat \cr
\noalign{\vskip4pt}
(\sv\omega \x \jhat\?') * \ihat &
(\sv\omega \x \jhat\?') * \jhat &
(\sv\omega \x \jhat\?') * \khat \cr
\noalign{\vskip4pt}
(\sv\omega \x \khat') * \ihat &
(\sv\omega \x \khat') * \jhat &
(\sv\omega \x \khat') * \khat \cr
}\right]
$$
Likewise, the transpose derivative will be
$$
\q{d}{dt}{{\rm R^T}} =
\left[
\matrix{
(\sv\omega \x \ihat') * \ihat &
(\sv\omega \x \jhat\?') * \ihat &
(\sv\omega \x \khat') * \ihat \cr
\noalign{\vskip4pt}
(\sv\omega \x \ihat') * \jhat &
(\sv\omega \x \jhat\?') * \jhat &
(\sv\omega \x \khat') * \jhat \cr
\noalign{\vskip4pt}
(\sv\omega \x \ihat') * \khat &
(\sv\omega \x \khat') * \khat &
(\sv\omega \x \jhat\?') * \khat \cr
}\right]
$$
Now, an interesting thing happens when we multiply inertial coordinates by the
time derivative of $\rm R$:
$$
\eqalign{
\q{d}{dt}{\rm R} * \!\,
\left[\matrix{x \cr y \cr z}\right]
&=
\left[
\matrix{
(\sv\omega \x \ihat') * \ihat &
(\sv\omega \x \ihat') * \jhat &
(\sv\omega \x \ihat') * \khat \cr
\noalign{\vskip4pt}
(\sv\omega \x \jhat\?') * \ihat &
(\sv\omega \x \jhat\?') * \jhat &
(\sv\omega \x \jhat\?') * \khat \cr
\noalign{\vskip4pt}
(\sv\omega \x \khat') * \ihat &
(\sv\omega \x \khat') * \jhat &
(\sv\omega \x \khat') * \khat \cr
}\middle]
\middle[\matrix{
x \cr
\noalign{\vskip4pt}
y \cr
\noalign{\vskip4pt}
z }\right]
\cr
\noalign{\vskip8pt}
&=
\matrix{
(\sv\omega \x \ihat') * x\?\ihat +
(\sv\omega \x \ihat') * y\?\jhat +
(\sv\omega \x \ihat') * z\?\khat \cr
\noalign{\vskip4pt}
(\sv\omega \x \jhat\?') * x\?\ihat +
(\sv\omega \x \jhat\?') * y\?\jhat +
(\sv\omega \x \jhat\?') * z\?\khat \cr
\noalign{\vskip4pt}
(\sv\omega \x \khat') * x\?\ihat +
(\sv\omega \x \khat') * y\?\jhat +
(\sv\omega \x \khat') * z\?\khat \cr
}
\cr
\noalign{\vskip8pt}
&=
\matrix{
(x\?\ihat \x \sv\omega) * \ihat' +
(y\?\jhat \x \sv\omega) * \ihat' +
(z\?\khat \x \sv\omega) * \ihat'\cr
\noalign{\vskip4pt}
(x\?\ihat \x \sv\omega) * \jhat\?' +
(y\?\jhat \x \sv\omega) * \jhat\?' +
(z\?\khat \x \sv\omega) * \jhat\?' \cr
\noalign{\vskip4pt}
(x\?\ihat \x \sv\omega) * \khat' +
(y\?\jhat \x \sv\omega) * \khat' +
(z\?\khat \x \sv\omega) * \khat' \cr
}
\cr
\noalign{\vskip8pt}
&=
\matrix{
\bigl((x\?\ihat + y\?\jhat + z\?\khat) \x \sv\omega\bigr) * \ihat' \cr
\noalign{\vskip4pt}
\bigl((x\?\ihat + y\?\jhat + z\?\khat) \x \sv\omega\bigr) * \jhat' \cr
\noalign{\vskip4pt}
\bigl((x\?\ihat + y\?\jhat + z\?\khat) \x \sv\omega\bigr) * \khat' \cr
}
\cr
\noalign{\vskip8pt}
&=
\matrix{
\(\sv r \x \sv\omega\) * \ihat' \cr
\noalign{\vskip4pt}
\(\sv r \x \sv\omega\) * \jhat' \cr
\noalign{\vskip4pt}
\(\sv r \x \sv\omega\) * \khat' \cr
}
}
$$
And so we see that this matrix multiplication has boiled down to
$$
\q{d}{dt}{\rm R} * \!\,
\left[\matrix{x \cr y \cr z}\right]
= {\rm R}\,(\sv r \x \sv\omega)
$$
By very similar algebra it can also be shown that
$$
\eqalign{
\q{d}{dt}{\rm R^T} *
\left[\matrix{x' \cr y' \cr z'}\right]
&=
\matrix{
\bigl(\sv\omega \x (x'\ihat' + y'\jhat\?' + z'\khat')\bigr) * \ihat \cr
\noalign{\vskip4pt}
\bigl(\sv\omega \x (x'\ihat' + y'\jhat\?' + z'\khat')\bigr) * \jhat \cr
\noalign{\vskip4pt}
\bigl(\sv\omega \x (x'\ihat' + y'\jhat\?' + z'\khat')\bigr) * \khat \cr
}
\cr
\noalign{\vskip8pt}
&=
\matrix{
\bigl(\sv\omega \x (x\?\ihat + y\?\jhat + z\?\khat)\bigr) * \ihat \cr
\noalign{\vskip4pt}
\bigl(\sv\omega \x (x\?\ihat + y\?\jhat + z\?\khat)\bigr) * \jhat \cr
\noalign{\vskip4pt}
\bigl(\sv\omega \x (x\?\ihat + y\?\jhat + z\?\khat)\bigr) * \khat \cr
}
\cr
\noalign{\vskip8pt}
&=
\matrix{
\(\sv\omega \x \sv r\) * \ihat \cr
\noalign{\vskip4pt}
\(\sv\omega \x \sv r\) * \jhat \cr
\noalign{\vskip4pt}
\(\sv\omega \x \sv r\) * \khat \cr
}
}
$$
As before, this can be more compactly written as
$$
\q{d}{dt}{\rm R^T}  *
\left[\matrix{x' \cr y' \cr z'}\right]
= \sv\omega \x \sv r
$$
With these results in hand we can now continue with our previous work:
$$
\eqalign{
\sv\tau
&=
\[\q{d}{dt}{\rm R^T}\]\sv L'
+ {\rm R^T}{\rm I}'\[\q{d}{dt}{\rm R}\]\sv\omega
+ {\rm I}\?\dot{\sv\omega}
\cr
&=
\sv\omega \x \sv L
+ {\rm R^T}{\rm I}'{\rm R}\,(\sv\omega \x \sv\omega)
+ {\rm I}\?\dot{\sv\omega}
\cr
\noalign{\vskip4pt}
&=
\sv\omega \x \sv L + {\rm I}\?\dot{\sv\omega}
}
$$
We now have
$$
{\rm I}\?\dot{\sv\omega} = \sv\tau - \sv\omega \x \sv L
$$
In order to invert the tensor of inertia we can make use of its form in the
body frame:
$$
{\rm R^T}{\rm I}'{\rm R}\,\dot{\sv\omega}
= \sv\tau - \sv\omega \x \sv L
$$
Our final result then immediately follows:
$$
\dot{\sv\omega}
= {\rm R^T}{{\rm I}'}^{-1}{\rm R}
\(\sv\tau - \sv\omega \x \sv L\)
$$
If only $\sv\omega$ and ${\rm I}'$ are known, and $\sv L$ is unknown expect
insofar as it is a function of these two, then it is sometimes useful to write
$$
\dot{\sv\omega}
= {\rm R^T}{{\rm I}'}^{-1}{\rm R}
\(
  \sv\tau
  -
  {\rm R^T}({\rm R}\?\sv\omega \x {\rm I}'{\rm R}\?\sv\omega\?)
\)
$$
The above is the most verbose possible form of this result. It has so many
terms because it functions in the inertial coordinate system and it assumes
that ${\rm I}'$ is not diagonalized. Let us simplify things by allowing
ourselves to work in body coordinates:
$$
\eqalign{
\dot{\sv\omega}
&= {\rm R^T}{{\rm I}'}^{-1}{\rm R}
\(\sv\tau - \sv\omega \x \sv L\)
\cr
\dot{\sv\omega}
&= {\rm R^T}{{\rm I}'}^{-1}
\({\rm R}\?\sv\tau - {\rm R}\?(\sv\omega \x \sv L)\)
\cr
{\rm I}'{\rm R}\,\dot{\sv\omega}
&= 
\({\rm R}\sv\tau - {\rm R}\?(\sv\omega \x \sv L)\)
\cr
{\rm I}'\dot{\sv\omega}'
&= 
\(\sv\tau' - \sv\omega'\!\, \x \sv L'\)
\cr
{\rm I}'\dot{\sv\omega}'
&= 
\(\sv\tau' - \sv\omega'\!\, \x {\rm I}'\sv\omega'\)
\cr
\dot{\sv\omega}'
&= 
{{\rm I}'}^{-1}\(\sv\tau' - \sv\omega'\!\, \x {\rm I}'\sv\omega'\)
}
$$
Let us then also assume that ${\rm I}'$ is diagonalized:
$$
{\rm I}' =
\left[
\matrix{
\lambda_1 & 0 & 0 \cr
0 & \lambda_2 & 0 \cr
0 & 0 & \lambda_3 \cr
}\right]
$$
In this case our vector equation for $\dot\sv\omega'$ will split cleanly into
three parts. These are called Euler's equations:
$$
\eqalign{
\dot\omega_x'
&= \q{\tau_x' - \omega_y'\omega_z'(\lambda_3 - \lambda_2)}{\lambda_1}
\cr
\noalign{\vskip6pt}
\dot\omega_y'
&= \q{\tau_y' - \omega_z'\omega_x'(\lambda_1 - \lambda_3)}{\lambda_2}
\cr
\noalign{\vskip6pt}
\dot\omega_z'
&= \q{\tau_z' - \omega_x'\omega_y'(\lambda_2 - \lambda_1)}{\lambda_3}
\cr
}
$$
Euler's equations are frequently written without the primes, and it is assumed
that the reader is aware that they must be evaluated in body coordinates:
$$
\eqalign{
\dot\omega_x
&= \q{\tau_x - \omega_y\omega_z(\lambda_3 - \lambda_2)}{\lambda_1}
\cr
\noalign{\vskip6pt}
\dot\omega_y
&= \q{\tau_y - \omega_z\omega_x(\lambda_1 - \lambda_3)}{\lambda_2}
\cr
\noalign{\vskip6pt}
\dot\omega_z
&= \q{\tau_z - \omega_x\omega_y(\lambda_2 - \lambda_1)}{\lambda_3}
\cr
}
$$
In order to compute $\dot\sv\omega$ in inertial coordinates one must map
$\sv\tau$ and $\sv\omega$ into body coordinates, compute the derivative
$\dot\sv\omega$ in body coordinates via Euler's equations, and then map the
result back into inertial coordinates.

\bye
