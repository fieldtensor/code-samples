\input ../paper.tex

\ignore{
\nopagenumbers
\pagesetup{1000}{3.5in}{4.5in}{0.11in}{0.11in}

\hsize=\pdfpagewidth
\vsize=\pdfpageheight
\advance\hsize by -0.22in
\advance\vsize by -0.22in

\tolerance=1000
}

\def\rotationmatrix{%
$$
\sv\Psi =
\[%
\matrix{
\vrule width 0pt depth 7pt
\phantom{+\omega_x}c + \omega_x\omega_x\(1 - c\) &
-\omega_zs + \omega_x\omega_y\(1-c\)  &
+\omega_ys + \omega_x\omega_z\(1-c\) 
\cr
\vrule width 0pt depth 7pt
+\omega_zs + \omega_y\omega_x\(1-c\?\)  &
\phantom{+\omega_x}c + \omega_y\omega_y\(1 - c\) &
-\omega_xs + \omega_y\omega_z\(1-c\) 
\cr
-\omega_ys + \omega_z\omega_x\(1-c\)  &
+\omega_xs + \omega_z\omega_y\(1-c\) &
\phantom{+\omega_x}c + \omega_z\omega_z\(1 - c\) \cr
}%
\]
$$
}

\papertitle{Rotation About a Vector}

\paperheading{Summary}

\noindent
The rotation $\sv R$ of the vector $\sv r$ through an angle $\psi$ about some
unit vector $\sv{\hat\omega}$ can be computed with
$$
\sv R(\sv r, \sv{\hat\omega}, \psi)
= \sin\psi*\[\sv{\hat\omega} \x \sv r\]
+ \cos\psi*\sv r
+ \sv{\hat\omega}\(1-\cos\psi\)\(\sv{\hat\omega} * \sv r\)
$$
The matrix form of this result is $$\sv R\(\sv r, \sv{\hat\omega}, \psi\) =
\sv\Psi\sv r$$ where $\sv\Psi$ is the rotation matrix \rotationmatrix Note that we
have used the shorthand $c = \cos\psi$ and $s = \sin\psi$.

\paperheading{Discovery}

\noindent We will consider the problem of taking any point in space $\sv r$
and rotating it through some angle $\psi$ about the unit vector
$\sv{\hat\omega}$. To begin, let us use label $\theta$ to denote the angle
between $\sv r$ and $\sv \omega$. We will split $\sv r$ into $\sv
r_{\parallel}$ and $\sv r_{\perp}$. We will form $\sv r_{\parallel}$ by
projecting $\sv r$ onto $\sv{\hat\omega}$ and we will form $\sv r_{\perp}$ by
taking whatever is left over after the projection
$$
\eqalign{
\sv r_{\parallel}
&= r \cos\theta * \sv{\hat\omega} \cr
&= \sv{\hat\omega}\(\sv{\hat\omega} * \sv r\) \cr
%
\vrule width 0pt height 15pt
\sv r_{\perp}
&= \sv r - \sv r_{\parallel} \cr
&= \sv r - \sv{\hat\omega}\(\sv{\hat\omega} * \sv r\)
}
$$
We now consider the plane which is perpendicular $\sv r_{\parallel}$ and drawn
at the tip of $\sv r_{\parallel}$. We wish to find basis vectors in that
plane. The vector $\sv r_{\perp}$ is already one such basis vector. We know
that our second basis vector $\sv\rho$ will be perpendicular to both $\sv
r_{\perp}$ and $\sv{\hat\omega}$. We can therefore find $\sv\rho$ as follows
$$
\sv\rho
= \sv{\hat\omega} \x \sv r_{\perp}
= \sv{\hat\omega} \x \(\sv r - \sv r_{\parallel}\)
= \sv{\hat\omega} \x \sv r
$$
Notice that $\sv\rho$ is conveniently of the same magnitude as $\sv
r_{\perp}$. To find where $\sv r$ goes when rotated about $\sv{\hat\omega}$ we
make a circular combination of $\sv r_\perp$ and $\sv\rho$, which we then
add back to $\sv r_\parallel$, as follows:
$$
\eqalign{
\sv R(\sv r, \sv{\hat\omega}, \psi)
&= \sv r_{\parallel} + \sv r_{\perp}\cos\psi + \sv\rho\sin\psi \cr
&= \sv{\hat\omega}\(\sv{\hat\omega} * \sv r\)
+ \cos\psi*\[\sv r - \sv{\hat\omega}\(\sv{\hat\omega} \cdot \sv r\)\]
+ \sin\psi*\[\sv{\hat\omega} \x \sv r\]
\cr
&= \sin\psi*\[\sv{\hat\omega} \x \sv r\]
+ \cos\psi*\sv r
+ \sv{\hat\omega}\(1-\cos\psi\)\(\sv{\hat\omega} * \sv r\)
}
$$
We give the component expansion of this result below. For the sake of
readability we will use $c$ and $s$ for $\cos\psi$ and $\sin\psi$.
\def\qdot{\omega_x r_x + \omega_y r_y + \omega_z r_z}
$$
\eqalign{
R_x &= s\(\omega_yr_z - \omega_zr_y\) + cr_x + \omega_x\(1 - c\)\(\qdot\) \cr
R_y &= s\(\omega_zr_x - \omega_xr_z\) + cr_y + \omega_y\(1 - c\)\(\qdot\) \cr
R_z &= s\(\omega_xr_y - \omega_yr_x\) + cr_z + \omega_z\(1 - c\)\(\qdot\) \cr
}
$$
It is clear that each of these components can be written as a linear
combination of $r_x$, $r_y$, and $r_z$. In order to see this explicitly we
first expand out all multiplications
$$
\mathcramp{0.9}
\eqalign{
R_x &=
s\omega_yr_z - s\omega_zr_y
+ cr_x
+ \omega_x\omega_xr_x + \omega_x\omega_yr_y + \omega_x\omega_zr_z
- c\omega_x\omega_xr_x - c\omega_x\omega_yr_y - c\omega_x\omega_zr_z
\cr
R_y &=
s\omega_zr_x
- s\omega_xr_z
+ cr_y
+ \omega_y\omega_xr_x + \omega_y\omega_yr_y + \omega_y\omega_zr_z
- c\omega_y\omega_xr_x - c\omega_y\omega_yr_y - c\omega_y\omega_zr_z \cr
R_z &=
s\omega_xr_y - s\omega_yr_x
+ cr_z
+ \omega_z\omega_xr_x + \omega_z\omega_yr_y + \omega_z\omega_zr_z
- c\omega_z\omega_xr_x - c\omega_z\omega_yr_y - c\omega_z\omega_zr_z
\cr
}
$$
Then we collect the terms in order to make the linear combinations stand out
$$
\eqalign{
R_x &=
r_x\(\omega_x\omega_x + c - c\?\omega_x\omega_x\)
+ r_y\(\omega_x\omega_y - c\?\omega_x\omega_y - s\?\omega_z\)
+ r_z\(\omega_x\omega_z - c\?\omega_x\omega_z + s\?\omega_y\)
\cr
%\scr{0pt}{6pt}
R_y &= r_x\(\omega_x\omega_y - c\?\omega_x\omega_y + s\?\omega_z\)
+ r_y\(\omega_y\omega_y + c - c\?\omega_y\omega_y\)
+ r_z\(\omega_y\omega_z - c\?\omega_y\omega_z - s\?\omega_x\)
\cr
%\scr{0pt}{6pt}
R_z &= r_x\(\omega_x\omega_z - c\?\omega_x\omega_z - s\?\omega_y\)
+ r_y\(\omega_y\omega_z - c\?\omega_y\omega_z + s\?\omega_x\)
+ r_z\(\omega_z\omega_z + c - c\?\omega_z\omega_z\)
}
$$
This result is best written in matrix form as $$\sv R(\sv r, \sv{\hat\omega},
\psi) = \sv\Psi\sv r$$ where $\sv\Psi$ is the rotation matrix \rotationmatrix

\bye
