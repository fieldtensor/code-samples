\input ../paper.tex

\def\ccr{\cr\noalign{\vskip3pt}}

\papertitle{Spherical Coordinates}

\paperheading{The Basis Vectors}

In the calculus of non-Cartesian coordinate systems we typically define the
basis vectors in terms of the gradients of the coordinates themselves. In
spherical coordinates this means that
$$
\eqalign{
\sv{\hat r} &= \norm\del r \cr
\sv{\hat \theta} &= \norm\del \theta \cr
\sv{\hat \varphi} &= \norm\del \phi
}
$$
The gradients themselves might not be of unit length. If we want basis vectors
of unit length we must explicitly perform a normalization, which is why we
have written $\norm \del r$ instead of just $\del r$ (though it just so
happens that $\del r$ comes out normalized by default).

The gradient $\del\theta$ will point in the direction of increasing $\theta$.
Defining $\sv{\hat\theta}$ in this way ensures that it points in the direction
of increasing $\theta$. This goes likewise for $r$ and $\sv{\hat r}$,
$\varphi$ and $\sv{\hat\varphi}$, and even $x$ and $\sv{\hat x}$ in Cartesian
coordinates. It should make immediate sense that $\sv{\hat r}$ points in the
direction of increasing $r$, but take a moment and try to convince yourself of
this for $\sv{\hat\theta}$ and $\sv{\hat\varphi}$ as well (for their
respective coordinates). This gradient-based definition of the basis vectors
works in other systems too, like for example cylindrical and polar. The basis
vectors used for curved space-times in general relativity are defined in a
similar manner, though the situation there is more complicated, and the basis
vectors are usually not normalized.

Let us try to find the basis vectors in the spherical system by taking the
gradients of $r$, $\theta$, and $\varphi$. The mapping from spherical
coordinates to Cartesian coordinates is
$$
\eqalign{
x &= r\cos\varphi\sin\theta \cr
y &= r\sin\varphi\sin\theta \cr
z &= r\cos\theta \cr
}
$$
The inverse mapping is
$$
r = \sqrt{x^2 + y^2 + z^2} \hskip18pt
\theta = \arctan\(\q{\sqrt{x^2 + y^2}}{z}\,\) \hskip18pt
\varphi = \arctan\(\q{y}{x}\)
$$
For brevity let us borrow some notation from the cylindrical system and write
$$
\rho = \sqrt{x^2 + y^2} = r\sin\theta
$$
The inverse mapping can then be written more compactly in terms of $\rho$ as
$$
r = \sqrt{x^2 + y^2 + z^2} \hskip18pt
\theta = \arctan\(\q{\rho}{z}\) \hskip18pt
\varphi = \arctan\(\q{y}{x}\)
$$
Now there is noting left to do but to manually turn the crank and produce the
gradients. We start by finding the gradient of $r$:
$$
\eqalignno{
\del r
&=
\q{\p r}{\p x}\,\ihat
+ \q{\p r}{\p y}\,\jhat
+ \q{\p r}{\p z}\,\khat
\ccr
&=
\q{x}{r}\,\ihat
+ \q{y}{r}\,\jhat
+ \q{z}{r}\,\khat
\ccr
&=
\q{r\cos\varphi\sin\theta}{r}\,\ihat
+ \q{r\sin\varphi\sin\theta}{r}\,\jhat
+ \q{r\cos\theta}{r}\,\khat
\ccr
&=
\cos\varphi\sin\theta\,\ihat
+ \sin\varphi\sin\theta\,\jhat
+ \cos\theta\,\khat
}
$$
Next we find gradient of $\theta$:
$$
\eqalignno{
\del\theta
&=
\q{\p \theta}{\p x}\,\ihat
+ \q{\p \theta}{\p y}\,\jhat
+ \q{\p \theta}{\p z}\,\khat
\ccr
&=
\q{\p}{\p x}\arctan\(\q{\rho}{z}\)\ihat
+ \q{\p}{\p y}\arctan\(\q{\rho}{z}\)\jhat
+ \q{\p}{\p z}\arctan\(\q{\rho}{z}\)\khat
\ccr
&=
\q{1}{1 + \rho^2/z^2}\q{\p}{\p x}\(\q{\rho}{z}\)\ihat
+ \q{1}{1 + \rho^2/z^2}\q{\p}{\p y}\(\q{\rho}{z}\)\jhat
+ \q{1}{1 + \rho^2/z^2}\q{\p}{\p z}\(\q{\rho}{z}\)\khat
\ccr
&=
\q{1}{1 + \rho^2/z^2}\(\q{x}{z\rho}\)\ihat
+ \q{1}{1 + \rho^2/z^2}\(\q{y}{z\rho}\)\jhat
- \q{1}{1 + \rho^2/z^2}\(\q{\rho}{z^2}\)\khat
\ccr
&=
\q{xz}{\rho r^2}\,\ihat
+ \q{yz}{\rho r^2}\,\jhat
- \q{\rho}{r^2}\,\khat
\ccr
&=
\q{r\cos\varphi\sin\theta*r\cos\theta}{r\sin\theta * r^2}\,\ihat
+ \q{r\sin\varphi\sin\theta*r\cos\theta}{r\sin\theta*r^2}\,\jhat
- \q{r\sin\theta}{r^2}\,\khat
\ccr
&=
\q{\cos\varphi\cos\theta}{r}\,\ihat
+ \q{\sin\varphi\cos\theta}{r}\,\jhat
- \q{\sin\theta}{r}\,\khat
}
$$
Finally we find the gradient of $\varphi$:
$$
\eqalignno{
\del\varphi
&=
\q{\p \varphi}{\p x}\,\ihat
+ \q{\p \varphi}{\p y}\,\jhat
+ \q{\p \varphi}{\p z}\,\khat
\ccr
&=
\q{\p}{\p x}\arctan\(\q{y}{x}\)\ihat
+ \q{\p}{\p y}\arctan\(\q{y}{x}\)\jhat
+ \q{\p}{\p z}\arctan\(\q{y}{x}\)\khat
\ccr
&=
\q{1}{1 + y^2/x^2}\q{\p}{\p x}\(\q{y}{x}\)\ihat
+ \q{1}{1 + y^2/x^2}\q{\p}{\p y}\(\q{y}{x}\)\jhat
\ccr
&=
- \q{1}{1 + y^2/x^2}\(\q{y}{x^2}\)\ihat
+ \q{1}{1 + y^2/x^2}\(\q{1}{x}\)\jhat
\ccr
&=
- \q{y}{x^2 + y^2}\,\ihat
+ \q{x}{x^2 + y^2}\,\jhat
\ccr
&=
- \q{y}{\rho}\,\ihat
+ \q{x}{\rho}\,\jhat
\ccr
&=
- \q{\sin\varphi\sin\theta}{r\sin\theta}\,\ihat
+ \q{\cos\varphi\sin\theta}{r\sin\theta}\,\jhat
\ccr
&=
- \q{\sin\varphi}{r}\,\ihat
+ \q{\cos\varphi}{r}\,\jhat
}
$$
In summary the gradients are:
$$
\eqalignno{
\del r &= 
\cos\varphi\sin\theta\,\ihat
+ \sin\varphi\sin\theta\,\jhat
+ \cos\theta\,\khat
\ccr
\noalign{\vskip3pt}
\del\theta &= 
\q{\cos\varphi\cos\theta}{r}\,\ihat
+\q{\sin\varphi\cos\theta}{r}\,\jhat
-\q{\sin\theta}{r}\,\khat
\ccr
\del\varphi &=
-\q{\sin\varphi}{r}\,\ihat
+\q{\cos\varphi}{r}\,\jhat
}
$$
As mentioned earlier we must normalize these gradients to get the associated
basis vectors:
$$
\eqalignno{
\sv{\hat r} &= 
\cos\varphi\sin\theta\,\ihat
+ \sin\varphi\sin\theta\,\jhat
+ \cos\theta\,\khat
\ccr
\sv{\hat\theta} &= 
\cos\varphi\cos\theta\,\ihat
+\sin\varphi\cos\theta\,\jhat
-\sin\theta\,\khat
\ccr
\sv{\hat\varphi} &=
-\sin\varphi\,\ihat
+\cos\varphi\,\jhat
}
$$
We could have perhaps found these vectors by circumscribing circles in a
sphere and working out the trigonometry, but it is nonetheless satisfying to
see that the formal gradient definitions really do work.

\paperheading{The Velocity}

Now that we have our basis vectors we can work out an expression for velocity
in spherical coordinates. The position vector in spherical coordinates is just
$\sv r = r\sv{\hat r}$. The velocity is therefore
$$
\eqalign{
\sv v
=
\q{d\sv r}{dt}
&=
\q{d}{dt}\Bigl[r\?\sv{\hat r}\Bigr]
\ccr
&=
\dot r\sv{\hat r}
+ r\q{d\sv{\hat r}}{dt}
\ccr
&=
\dot r\sv{\hat r}
+ \q{d}{dt}\Bigl[
\cos\varphi\sin\theta\,\ihat
+ \sin\varphi\sin\theta\,\jhat
+ \cos\theta\,\khat
\Bigr]
\ccr
&=
\dot r\sv{\hat r}
+ \dot\theta\Bigl[
\cos\varphi\cos\theta\,\ihat
+ \sin\varphi\cos\theta\,\jhat
- \sin\theta\,\khat
\Bigr]
\cr
&
\hskip30pt
+ \dot\varphi\Bigl[
- \sin\varphi\sin\theta\,\ihat
+ \cos\varphi\sin\theta\,\jhat
\Bigr]
\ccr
&=
\dot r\?\sv{\hat r}
+ r\?\dot\theta\?\sv{\hat\theta}
+ r\sin\theta\,\dot\varphi\?\sv{\hat\varphi}
}
$$
In summary this is
$$
\sv v
=
\dot r\?\sv{\hat r}
+ r\?\dot\theta\?\sv{\hat\theta}
+ r\sin\theta\,\dot\varphi\?\sv{\hat\varphi}
$$
Before we go on take note that that the velocity was especially easy to
formulate in spherical coordinates because the position vector was just $\sv r
= r\?\sv{\hat r}$. In cylindrical coordinates the position vector is $\sv r =
\rho\sv{\hat\rho} + z\?\sv{\hat z}$, and in other systems the position might
be a yet more complicated mixture of the basis vectors and their coordinates.

\paperheading{The Gradient}

Consider a scalar function $\psi$ of euclidean space and some associated
gradient $\del\psi$. Recall that the projection of some vector $\sv b$ onto
some unit vector $\sv{\hat a}$ is defined as $\sv{\hat a}\(\sv{\hat a} * \sv
b\)$. We can use this projection method to split $\del\psi$ into its component
parts on the spherical basis vectors:
$$
\del \psi
=
\Bigl(\del\psi * \sv{\hat r}\Bigr)\,\sv{\hat r}
+ \Bigl(\del\psi * \sv{\hat\theta}\Bigr)\,\sv{\hat\theta}
+ \Bigl(\del\psi * \sv{\hat\varphi}\Bigr)\,\sv{\hat\varphi}
$$
For brevity let us write this as 
$$
\del \psi
=
G_r\,\sv{\hat r}
+ G_\theta\,\sv{\hat\theta}
+ G_\varphi\,\sv{\hat\varphi}
$$
where $G_r = \del\psi * \sv{\hat r}$, $G_\theta = \del\psi * \sv{\hat\theta}$,
and $G_\varphi = \del\psi * \sv{\hat\varphi}$. The values of $G_r$,
$G_\theta$, and $G_\varphi$ are still unknown to us. It is the goal of this
section to figure out what they should be. We will find these terms by moving
down some path in space and computing $d\psi/dt$ as we go. We will make this
computation of $d\psi/dt$ twice, each time using a different method. We will
then say that both methods must give the same answer, because physically there
is only one $d\psi/dt$ down any given path. Formally equating these two
methods will give us the information we seek about $G_r$, $G_\theta$, and
$G_\varphi$.

For the first method consider moving along some path in Cartesian coordinates
and measuring $\psi$ as you go. We can take time derivative of $\psi$ over our
path in the usual way:
$$
\q{d\psi}{dt}
= \q{d}{dt}\psi(x, y, z)
= \q{\p \psi}{\p x}\q{dx}{dt}
+ \q{\p \psi}{\p y}\q{dy}{dt}
+ \q{\p \psi}{\p z}\q{dz}{dt}
= \del\psi * \sv v
$$
In summary we have:
$$
\q{d\psi}{dt} = \del\psi * \sv v
$$
In the previous section we found a spherical expression for the velocity $\sv
v$. We can apply this to our expression for $d\psi/dt$:
$$
\eqalign{
\q{d\psi}{dt}
= \del\psi * \sv v
&= \(G_r\,\sv{\hat r}
+ G_\theta\,\sv{\hat\theta}
+ G_\varphi\,\sv{\hat\varphi}\)
*
\(\dot r\?\sv{\hat r}
+ r\?\dot\theta\?\sv{\hat\theta}
+ r\sin\theta\,\dot\varphi\?\sv{\hat\varphi}\)
\cr
&=
\dot r \? G_r
+ r\?\dot\theta \? G_\theta
+ r\sin\theta\,\dot\varphi \? G_\varphi
}
$$
In summary we have
$$
\q{d\psi}{dt}
=
\dot r \? G_r
+ r\?\dot\theta \? G_\theta
+ r\sin\theta\,\dot\varphi \? G_\varphi
$$
That concludes the first method. Now for the second method. If we are working
in spherical coordinates then presumably we have an expression for $\psi$ in
terms of $r$, $\theta$, and $\varphi$. We can use spherical coordinates to
express both $\psi$ and the path along which we measure $\psi$. When we do so
we find that
$$
\eqalign{
\q{d\psi}{dt}
&=
\q{d}{dt}\psi(r, \theta, \varphi)
\ccr
&=
\q{\p \psi}{\p r}\q{dr}{dt}
+ \q{\p \psi}{\p\theta}\q{d\theta}{dt}
+ \q{\p \psi}{\p\varphi}\q{d\varphi}{dt}
\ccr
&=
\q{\p \psi}{\p r}\dot r
+ \q{\p \psi}{\p\theta}\dot\theta
+ \q{\p \psi}{\p\varphi}\dot\varphi
}
$$
These two lines of calculus are hardly necessary to summarize, but let us
write a summary anyway:
$$
\q{d\psi}{dt}
=
\q{\p \psi}{\p r}\dot r
+ \q{\p \psi}{\p\theta}\dot\theta
+ \q{\p \psi}{\p\varphi}\dot\varphi
$$
That concludes the second method. The form of the answer in this second method
is different from the form in the first because it does not involve the $G$
terms, nor does it involve the expression for $\sv v$ in spherical
coordinates. Rather, this new form looks almost like a dot product between
$$\(\q{\p \psi}{dr}, \q{\p \psi}{d\theta}, \q{\p \psi}{d\psi}\) \hskip6pt {\rm
and} \hskip6pt \(\dot r, \dot\theta, \dot\varphi\)$$ These look almost like a
gradient vector and a velocity vector. Yet the left one is not actually a
gradient vector, and the right one is plainly not a velocity vector. Some of
its terms do not even have units of meters per second!

We have now computed $d\psi/dt$ in two
separate ways. Both methods must give the same answer, numerically speaking,
and so we must have
$$
\dot r \? G_r
+ r\?\dot\theta \? G_\theta
+ r\sin\theta\,\dot\varphi \? G_\varphi
=
\q{\p \psi}{\p r}\dot r
+ \q{\p \psi}{\p\theta}\dot\theta
+ \q{\p \psi}{\p\varphi}\dot\varphi
$$
After some shuffling and factoring we can express this relation as
$$
\dot r\(G_r
- \q{\p \psi}{\p r}\)
+ \dot\theta\(r \? G_\theta
- \q{\p \psi}{\p\theta}\)
+ \dot\varphi\(r\sin\theta\, \? G_\varphi
- \q{\p \psi}{\p\varphi}\)
=
0
$$
The derivatives $\dot r$, $\dot\theta$, and $\dot\varphi$ are characteristic
of the path along which we are measuring $d\psi/dt$. For example, if we are
measuring $d\psi/dt$ along a path moving radially outward then we have only
$\dot r$, while $\dot\theta$ and $\dot\varphi$ vanish. If we are measuring
$d\psi/dt$ while moving in an azimuthal circle then perhaps we have only
$\dot\varphi$, while $\dot r$ and $\dot\theta$ vanish. The point is that the
above sum must come out to $0$ for any and all paths, and thus we must have
$$
G_r - \q{\p \psi}{\p r} = 0
\hskip18pt
r \? G_\theta - \q{\p \psi}{\p\theta} = 0
\hskip18pt
r\sin\theta\, \? G_\varphi - \q{\p \psi}{\p\varphi} = 0
$$
These relations let us compute the $G$ values without too much more effort:
$$
G_r = \q{\p \psi}{\p r}
\hskip18pt
G_\theta = \q1r \q{\p \psi}{\p\theta}
\hskip18pt
G_\varphi = \q{1}{r\sin\theta}\q{\p \psi}{\p\varphi}
$$
The gradient in spherical coordinates is therefore
$$
\del\psi
=
\q{\p \psi}{\p r}\,\sv{\hat r}
+ \q1r \q{\p \psi}{\p\theta}\,\sv{\hat\theta}
+ \q{1}{r\sin\theta}\q{\p \psi}{\p\varphi}\,\sv{\hat\varphi}
$$

\bye
