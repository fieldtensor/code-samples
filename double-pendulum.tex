\input ../paper.tex

\papertitle{The Double Pendulum}

\paperheading{Summary}

\noindent
Consider a double pendulum composed of masses $m_1$ and $m_2$. Let us say that
mass $m_1$ has polar coordinates $(\alpha, \theta)$ with respect to the
origin, and that mass $m_2$ has polar coordinates $(\beta, \phi)$ with respect
to mass $m_1$.

\vskip\baselineskip

\noindent The system of differential equations governing the motion of double
pendulum is as follows (note that we use the shorthand $M = m_1 + m_2$).
$$
\eqalign{
\[M\alpha^2\]\ddot\theta
+\[\alpha\beta\cos(\theta - \phi)\]\ddot\phi
&=
-m_2\alpha\beta{\dot\phi}^2\sin(\theta - \phi)
-Mg\alpha\cos\theta
\cr
\[m_2\alpha\beta\cos(\theta - \phi)\]\ddot\theta
+\[m_2\beta^2\]\ddot\phi
&=
m_2\alpha\beta\dot\theta^2\sin(\theta - \phi)
-m_2g\beta\cos\phi
}
$$

\noindent From the above equations can derive the separate generalized forces
acting on each mass. These forces are complicated to such a degree that they
are beyond intuitive human understanding. Nonetheless, we give them here:
$$
\eqalign{
\ddot\theta &=
\biggl[
\(\mathop{-}m_2\alpha\beta{\dot\phi}^2\sin(\theta - \phi) - Mg\alpha\cos\theta\)
\(m_2\beta^2\)
\cr
&\phantom{=}\,-
\(m_2\alpha\beta\cos(\theta - \phi)\)
\(m_2\alpha\beta\dot\theta^2\sin(\theta - \phi) - m_2g\beta\cos\phi\)
\biggr]\cr
&\phantom{=}\,* {1 / \[
\(M\alpha^2\)\(m_2\beta^2\)
-\(m_2\alpha\beta\cos(\theta - \phi)\)\(m_2\alpha\beta\cos(\theta - \phi)\)
\]}
\cr
\noalign{\medskip}
\ddot\phi &=
\biggl[
\(M\alpha^2\)
\(m_2\alpha\beta\dot\theta^2\sin(\theta - \phi) - m_2g\beta\cos\phi\)
\cr
&\phantom{=}\,-
\(\mathop{-}m_2\alpha\beta{\dot\phi}^2\sin(\theta - \phi) - Mg\alpha\cos\theta\)
\(m_2\alpha\beta\cos(\theta - \phi)\)
\biggr]\cr
&\phantom{=}\,* {1 / \[
\(M\alpha^2\)
\(m_2\beta^2\)
-\(m_2\alpha\beta\cos(\theta - \phi)\)
\(m_2\alpha\beta\cos(\theta - \phi)\)
\]}
}
$$

\paperheading{Discovery}

Consider a double pendulum composed of masses $m_1$ and $m_2$. Let us say that
mass $m_1$ has polar coordinates $(\alpha, \theta)$ with respect to the
origin, and that mass $m_2$ has polar coordinates $(\beta, \phi)$ with respect
to mass $m_1$.  Our goal is to find a formula for the forces on each body,
such that we can numerically integrate the forces and produce a computer
simulation of a swinging double-pendulum.

To begin, we say that the location of the first body is
$$
\eqalign{
x_1 &= \alpha \cos(\theta) \cr
y_1 &= \alpha \sin(\theta) \cr
}
$$
Next we say that the location of the second body is
$$
\eqalign{
x_2
&= x_1 + \beta \cos\phi \cr
&= \alpha \cos\theta + \beta \cos\phi
\cr
\noalign{\medskip}
y_2
&= y_1 + \beta \sin\phi \cr
&= \alpha \sin\theta + \beta \sin\phi \cr
}
$$
The squared velocity of the first body is then
$$
\cs v_1^2 = \alpha^2\dot\theta^2
$$
and the squared velocity of the second body is
$$
\eqalign{
\dot\cs x_1^2
&=
\(
-\alpha\dot\theta\sin\theta
-\beta\dot\phi\sin\phi
\)
^2
\cr
&=
\alpha^2\dot\theta^2\sin^2\theta
+2\alpha\beta\dot\theta\dot\phi\sin\theta\sin\phi
+\beta^2{\dot\phi}^2\sin^2\phi
\cr
\noalign{\medskip}
\dot\cs y_1^2
&=
\(
+\alpha\dot\theta\cos\theta
+\beta\dot\phi\cos\phi
\)^2
\cr
&=
\alpha^2\dot\theta^2\cos^2\theta
+2\alpha\beta\dot\theta\dot\phi\cos\theta\cos\phi
+\beta^2{\dot\phi}^2\cos^2\phi
\cr
\noalign{\vskip 9pt plus 2pt minus 2pt}
\cs v_2^2
&=
\dot\cs x_1^2 + \dot\cs y_1^2
\cr
&=
\alpha^2{\dot\theta^2}
+ \beta^2{\dot\phi^2}
+ 2\alpha\beta\dot\theta\dot\phi
\[
\sin\theta\sin\phi
+\cos\theta\cos\phi
\]
\cr
&=
\alpha^2{\dot\theta^2}
+ \beta^2{\dot\phi^2}
+ 2\alpha\beta\dot\theta\dot\phi\cos(\theta - \phi)
}
$$
The kinetic energy of the system can then be written as follows. Note that we
use the shorthand $M = m_1 + m_2$
$$
\eqalign{
T &=
\q12m_1\alpha^2\dot\theta^2
+ \q12m_2\alpha^2\dot\theta^2
+ \q12m_2\beta^2{\dot\phi}^2
+ m_2\alpha\beta\dot\theta\dot\phi\cos(\theta - \phi)
\cr
\noalign{\smallskip}
&=
\q12M\alpha^2\dot\theta^2
+ \q12m_2\beta^2{\dot\phi}^2
+ m_2\alpha\beta\dot\theta\dot\phi\cos(\theta - \phi)
}
$$

\noindent
The total potential energy of the system is just the usual
gravitational potential 
$$
\eqalign{
U
&=
m_1g\alpha\sin\theta + m_2g\[\alpha\sin\theta + \beta\sin\phi\]
\cr
\noalign{\smallskip}
&=
Mg\alpha\sin\theta + m_2g\beta\sin\phi
}
$$
The Lagrangian of the system is then simply
$$
{\cal L} = T - U
$$
The system's generalized momenta are then
$$
\eqalign{
p_\theta
&=
\q{\p {\cal L}}{\p \dot\theta}
=
M\alpha^2\dot\theta
+ m_2\alpha\beta\dot\phi\cos(\theta - \phi)
\cr
\noalign{\smallskip}
p_\phi
&=
\q{\p {\cal L}}{\p \dot\phi}
=
m_2\beta^2\dot\phi
+ m_2\alpha\beta\dot\theta\cos(\theta - \phi)
}
$$
and the system's generalized forces are
$$
\eqalign{
\q{\p {\cal L}}{\p \theta}
&=
\mathop{-}m_2\alpha\beta\dot\theta\dot\phi
\sin(\theta - \phi)
-Mg\alpha\cos\theta
\cr
\noalign{\smallskip}
\q{\p {\cal L}}{\p \phi}
&=
\mathop{+}m_2\alpha\beta\dot\theta\dot\phi
\sin(\theta - \phi)
-m_2g\beta\cos\phi
}
$$
In order to apply the Euler-Lagrange equations we must differentiate the
momenta with respect to time
$$
\eqalign{
\dot p_\theta
&=
M\alpha^2\ddot\theta
+ m_2\alpha\beta\ddot\phi\cos(\theta - \phi)
- m_2\alpha\beta\dot\phi\sin(\theta - \phi)
\[\dot\theta - \dot\phi\]
\cr
\dot p_\phi
&=
m_2\beta^2\ddot\phi
+ m_2\alpha\beta\ddot\theta\cos(\theta - \phi)
- m_2\alpha\beta\dot\theta\sin(\theta - \phi)
\[\dot\theta - \dot\phi\]
}
$$
Recall that the Euler-Lagrange equations for this system are
$$
\q{\p {\cal L}}{\p \theta} = \q{d}{dt}\mathinner{\Bigl[{\p {\cal L}}/{\p\dot\theta}\Bigr]}
\hskip25pt
\q{\p {\cal L}}{\p \phi} = \q{d}{dt}\mathinner{\Bigl[{\p {\cal L}}/{\p\dot\phi}\Bigr]}
$$
We have already calculated all of these terms, and so we can immediately
proceed
$$
\eqalign{
\mathop{-}m_2\alpha\beta\dot\theta\dot\phi
\sin(\theta - \phi)
-Mg\alpha\cos\theta
&=
M\alpha\ddot\theta + m_2\alpha\beta\ddot\phi\cos(\theta - \phi)
\cr
&\phantom{=}\mskip8mu
\mathop{-}m_2\alpha\beta\dot\phi\sin(\theta - \phi)
\[\dot\theta - \dot\phi\]
\cr
m_2\alpha\beta\dot\theta\dot\phi
\sin(\theta - \phi)
-m_2g\beta\cos\phi
&=
m_2\beta\ddot\phi
+ m_2\alpha\beta\ddot\theta\cos(\theta - \phi)
\cr
&\phantom{=}\mskip8mu
\mathop{-} m_2\alpha\beta\dot\theta\sin(\theta - \phi)
\[\dot\theta - \dot\phi\]
}
$$
Thankfully, some simplification here is possible
$$
\eqalign{
\mathop{-}Mg\alpha\cos\theta
&=
M\alpha^2\ddot\theta
+ m_2\alpha\beta\ddot\phi\cos(\theta - \phi)
+ m_2\alpha\beta{\dot\phi}^2\sin(\theta - \phi)
\cr
\mathop{-}m_2g\beta\cos\phi
&=
m_2\beta^2\ddot\phi
+ m_2\alpha\beta\ddot\theta\cos(\theta - \phi)
- m_2\alpha\beta\dot\theta^2\sin(\theta - \phi)
}
$$
We can view the above result as a pair of linear equations in $\ddot\theta$
and $\ddot\phi$.
$$
\eqalign{
\[M\alpha^2\]\ddot\theta
+\[m_2\alpha\beta\cos(\theta - \phi)\]\ddot\phi
&=
\mathop{-}m_2\alpha\beta{\dot\phi}^2\sin(\theta - \phi)
-Mg\alpha\cos\theta
\cr
\[m_2\alpha\beta\cos(\theta - \phi)\]\ddot\theta
+\[m_2\beta^2\]\ddot\phi
&=
m_2\alpha\beta\dot\theta^2\sin(\theta - \phi)
-m_2g\beta\cos\phi
}
$$
These equations can be solved for $\ddot\theta$ and $\ddot\phi$ by using
Cramer's rule.
$$
\eqalign{
\ddot\theta &=
\biggl[
\(\mathop{-}m_2\alpha\beta{\dot\phi}^2\sin(\theta - \phi) - Mg\alpha\cos\theta\)
\(m_2\beta^2\)
\cr
&\phantom{=}\,-
\(m_2\alpha\beta\cos(\theta - \phi)\)
\(m_2\alpha\beta\dot\theta^2\sin(\theta - \phi) - m_2g\beta\cos\phi\)
\biggr]\cr
&\phantom{=}\,* {1 / \[
\(M\alpha^2\)\(m_2\beta^2\)
-\(m_2\alpha\beta\cos(\theta - \phi)\)\(m_2\alpha\beta\cos(\theta - \phi)\)
\]}
\cr
\noalign{\medskip}
\ddot\phi &=
\biggl[
\(M\alpha^2\)
\(m_2\alpha\beta\dot\theta^2\sin(\theta - \phi) - m_2g\beta\cos\phi\)
\cr
&\phantom{=}\,-
\(\mathop{-}m_2\alpha\beta{\dot\phi}^2\sin(\theta - \phi) - Mg\alpha\cos\theta\)
\(m_2\alpha\beta\cos(\theta - \phi)\)
\biggr]\cr
&\phantom{=}\,* {1 / \[
\(M\alpha^2\)
\(m_2\beta^2\)
-\(m_2\alpha\beta\cos(\theta - \phi)\)
\(m_2\alpha\beta\cos(\theta - \phi)\)
\]}
}
$$

\noindent These solutions via Cramer's rule can easily be plugged into a
computer simulation and numerically integrated in order to show a
visualization of the chaotic motion of a double pendulum, a chaos which arises
from the tremendous complexity of the equations which we have derived here in
this paper.

\bye
