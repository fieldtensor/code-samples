\input ../paper.tex

\papertitle{The Transport Theorem}

\paperheading{Summary}

\noindent Given a fluid motion described by
$$
\sv u(\sv\phi(\sv r, t),t) = \q{\p \sv\phi}{\p t}
$$
the transport theorem tell us that
$$
\q{\p}{\p t} \iiint_{V_\phi} f(x,y,z,t) \; d^3V
= \iiint_{V_\phi} \q{Df}{Dt} + f \[\nabla * \sv u\] \; d^3V
$$

\paperheading{Discovery}

\noindent We will observe the motion of a fluid particle which starts at $\sv
r$, the future position of which we shall denote as $\sv\phi(\sv{r},t)$. We
will then use this function to define a vector field $\sv u$ representing the
fluid's velocity. This velocity field will obey the identity
$$
\sv u(\sv\phi(\sv r, t),t) = \q{\p \sv\phi}{\p t}
$$
We can observe a cluster of such fluid particles all at their initial
positions. We will say that this cluster makes a volume $V_0$. After some time
$t$ every fluid particle moves as per $\sv\phi(\sv{r},t)$ and the volume $V_0$
is transformed into $V_\phi$. Now, we wish to find the time derivative of some
function $f(x,y,z,t)$ over the volume $V_\phi$ as the volume moves:
$$
\q{\p}{\p t} \iiint_{V_\phi} f(x,y,z,t) \; d^3V
$$
In order to do this we will make use of the Jacobian determinant $J$ of
$\sv\phi$, for some fixed time $t$:
$$
J(x,y,z,t) = \det\left[
\dmatrix{
\noalign{\kern4pt}
\q{\p \phi_x}{\p x} & \q{\p \phi_x}{\p y} & \q{\p \phi_x}{\p z} \cr
\noalign{\kern4pt}
\q{\p \phi_y}{\p x} & \q{\p \phi_y}{\p y} & \q{\p \phi_y}{\p z} \cr
\noalign{\kern4pt}
\q{\p \phi_z}{\p x} & \q{\p \phi_z}{\p y} & \q{\p \phi_z}{\p z} \cr
\noalign{\kern4pt}
}
\right]
$$
This allows us to change our integral to
$$
\eqalign{
\cdots\:
&=\q{\p}{\p t} \iiint_{V_\phi} f(x,y,z,t) \; d^3V
\cr
&= \q{\p}{\p t} \iiint_{V_0} f(\sv\phi,t) * J \;\? d^3V
\cr
&= \iiint_{V_0}
\q{\p}{\p t}\Bigl[f(\sv\phi,t)\Bigr] \, J
+ f(\sv\phi,t) * \q{\p J}{\p t}
\;\? d^3V
\cr
&= \iiint_{V_0}
\[\nabla f(\sv\phi, t) * \q{\p\sv\phi}{\p t}
+ \q{\p f}{\p t} \circ (\sv\phi, t)\] J
+ f(\sv\phi,t) * \q{\p J}{\p t}
\;\? d^3V
\cr
&= \iiint_{V_0}
\[\nabla f * \sv u \?
+ \q{\p f}{\p t}\] \circ (\sv\phi, t) * J
+ f(\sv\phi,t) * \q{\p J}{\p t}
\;\? d^3V
\cr
}
$$
In order to proceed we must now find the time-derivative of $J$. First, let us
expand $J$ out in full:
\prot\def\detdiff#1#2{\q{\p\sv\phi_#1}{\p #2}}
$$
\eqalign{
J =
&+
\detdiff xx \detdiff yy \detdiff zz -
\detdiff xx \detdiff yz \detdiff zy
\cr
&+
\detdiff xy \detdiff yz \detdiff zx -
\detdiff xy \detdiff yx \detdiff zz
\cr
&+
\detdiff xz \detdiff yx \detdiff zy -
\detdiff xz \detdiff yy \detdiff zx
\cr
}
$$
Now, observe that taking a time derivative of any of the sub-terms in the
above will yield a result like
\prot\def\timediff#1#2{
\q{\p u_#1}{\p x}\q{\p \phi_x}{\p #2}
+\q{\p u_#1}{\p y}\q{\p \phi_y}{\p #2}
+\q{\p u_#1}{\p z}\q{\p \phi_z}{\p #2}}
$$
\q{\p}{\p t}\detdiff xy
= \q{\p}{\p y}\detdiff xt
= \q{\p}{\p y}\Bigl[u_y(\phi(\sv r, t), t)\Bigr]
= \timediff xy
$$
We can now take the time-derivative of $J$
$$
\eqalignno{
\q{\p J}{\p t} =
&+\[\timediff xx\] \detdiff yy \detdiff zz \cr
&+\detdiff xx \[\timediff yy\] \detdiff zz \cr
&+\detdiff xx \detdiff yy \[\timediff zz\] \cr
&-\[\timediff xx\] \detdiff yz \detdiff zy \cr
&-\detdiff xx \[\timediff yz\] \detdiff zy \cr
&-\detdiff xx \detdiff yz \[\timediff zy\] \cr
&+\[\timediff xy\] \detdiff yz \detdiff zx \cr
&+\detdiff xy \[\timediff yz\] \detdiff zx \cr
&+\detdiff xy \detdiff yz \[\timediff zx\] \cr
&-\[\timediff xy\] \detdiff yx \detdiff zz \cr
&-\detdiff xy \[\timediff yx\] \detdiff zz \cr
&-\detdiff xy \detdiff yx \[\timediff zz\] \cr
&+\[\timediff xz\] \detdiff yx \detdiff zy \cr
&+\detdiff xz \[\timediff yx\] \detdiff zy \cr
&+\detdiff xz \detdiff yx \[\timediff zy\] \cr
&-\[\timediff xz\] \detdiff yy \detdiff zx \cr
&-\detdiff xz \[\timediff yy\] \detdiff zx \cr
&-\detdiff xz \detdiff yy \[\timediff zx\] \cr
}
$$
After 12 cancellations we are left with
\prot\def\udiff#1#2{\q{\p u_#1}{\p #2}}
$$
\eqalign{
\q{\p J}{\p t} =
&
+\udiff xx \detdiff xx \detdiff yy \detdiff zz
+\udiff yy \detdiff xx \detdiff yy \detdiff zz
+\udiff zz \detdiff xx \detdiff yy \detdiff zz \cr
&
-\udiff xx \detdiff xx \detdiff yz \detdiff zy
-\udiff yy \detdiff xx \detdiff yz \detdiff zy
-\udiff zz \detdiff xx \detdiff yz \detdiff zy \cr
&
+\udiff xx \detdiff xy \detdiff yz \detdiff zx
+\udiff yy \detdiff xy \detdiff yz \detdiff zx
+\udiff zz \detdiff xy \detdiff yz \detdiff zx \cr
&
-\udiff xx \detdiff xy \detdiff yx \detdiff zz
-\udiff yy \detdiff xy \detdiff yx \detdiff zz
-\udiff zz \detdiff xy \detdiff yx \detdiff zz \cr
&
+\udiff xx \detdiff xz \detdiff yx \detdiff zy
+\udiff yy \detdiff xz \detdiff yx \detdiff zy
+\udiff zz \detdiff xz \detdiff yx \detdiff zy \cr
&
-\udiff xx \detdiff xz \detdiff yy \detdiff zx
-\udiff yy \detdiff xz \detdiff yy \detdiff zx
-\udiff zz \detdiff xz \detdiff yy \detdiff zx \cr
}
$$
We can then group these terms on derivatives of $\sv u$
$$
\eqalignno{
\q{\p J}{\p t} =
+\udiff xx
\biggl[
&
+\detdiff xx \detdiff yy \detdiff zz
-\detdiff xx \detdiff yz \detdiff zy \cr
&
+\detdiff xy \detdiff yz \detdiff zx
-\detdiff xy \detdiff yx \detdiff zz \cr
&
+\detdiff xz \detdiff yx \detdiff zy
-\detdiff xz \detdiff yy \detdiff zx
\biggr]
\cr
+\udiff yy
\biggl[
&
+\detdiff xx \detdiff yy \detdiff zz
-\detdiff xx \detdiff yz \detdiff zy \cr
&
+\detdiff xy \detdiff yz \detdiff zx
-\detdiff xy \detdiff yx \detdiff zz \cr
&
+\detdiff xz \detdiff yx \detdiff zy
-\detdiff xz \detdiff yy \detdiff zx
\biggr]
\cr
+\udiff zz
\biggl[
&
+\detdiff xx \detdiff yy \detdiff zz
-\detdiff xx \detdiff yz \detdiff zy \cr
&
+\detdiff xy \detdiff yz \detdiff zx
-\detdiff xy \detdiff yx \detdiff zz \cr
&
+\detdiff xz \detdiff yx \detdiff zy
-\detdiff xz \detdiff yy \detdiff zx
\biggr]
\cr
}
$$
We appear to have been left with Jacobian determinants, and so we can now
write
$$
\eqalign{
\q{\p J}{\p t} &= \udiff xx J + \udiff yy J + \udiff zz J \cr
&= \[\udiff xx + \udiff yy + \udiff zz\] J \cr
&= \Bigl[\(\nabla * \sv u\) \circ (\sv\phi, t)\Bigr] * J \cr
}
$$
We can then plug this into our integral from before
$$
\eqalign{
\cdots\:
&=\iiint_{V_0}
\[\nabla f * \sv u \?
+ \q{\p f}{\p t}\] \circ (\sv\phi, t) * J
+ f(\sv\phi,t) * \q{\p J}{\p t}
\; d^3V
\cr
&= \iiint_{V_0}
\[\nabla f * \sv u \?
+ \q{\p f}{\p t}\] \circ (\sv\phi, t) * J
+ f(\sv\phi,t) *
\bigl(\[\nabla * \sv u\] \circ (\sv\phi, t)\bigr) * J
\; d^3V
\cr
&= \iiint_{V_0}
\[\nabla f * \sv u \?
+ \q{\p f}{\p t}
+ f\[\nabla * \sv u\]\]  \circ (\sv\phi, t) * J
\; d^3V
\cr
&= \iiint_{V_\phi}
\nabla f * \sv u \?
+ \q{\p f}{\p t}
+ f \[\nabla * \sv u\]
\; d^3V
\cr
&= \iiint_{V_\phi}
\q{Df}{Dt}
+ f \[\nabla * \sv u\]
\; d^3V
\cr
}
$$
And so we have our final result:
$$
\q{\p}{\p t} \iiint_{V_\phi} f(x,y,z,t) \; d^3V
= \iiint_{V_\phi} \q{Df}{Dt} + f \[\nabla * \sv u\] \; d^3V
$$

\paperheading{Incompressible Fluids}

\noindent
If we set $f(x,y,z,t)=1$ then the integral in the transport theorem will
compute a fluid region's volume as it moves. If we know the fluid to be
incompressible then the time-derivative of this volume integral should be 0,
because an incompressible region will be able to deform, but never to change
grow or shrink in volume. Under this circumstance the transport theorem will
tell us that
$$
0 = \iiint_{V_\phi} \nabla * \sv u \;\, d^3V
$$
Thus we know that incompressible fluids have no velocity divergence
$$
\nabla * \sv u = 0
$$

\paperheading{Mass Flow}

\noindent We know that volumes in all fluids will maintain a constant mass as
they move, regardless of the type of fluid. The divergence theorem can thus
give us some insight into how a fluid's mass density $\rho$ evolves over time
$$
0
= \q{\p}{\p t} \iiint_{V_\phi} \rho \;\, d^3V
= \iiint_{V_\phi}
\nabla \rho * \sv u \?
+ \q{\p \rho}{\p t}
+ \rho \[\nabla * \sv u\]
\; d^3V
$$
We thus know that
$$
\q{\p \rho}{\p t} = \(-\nabla \rho\) * \sv u \? - \rho \[\nabla * \sv u\]
$$

\bye
