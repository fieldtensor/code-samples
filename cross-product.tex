\input ../paper.tex

\papertitleND{The Cross Product}

\paperheading{Summary}

\noindent The cross product of $\sv a$ and $\sv b$ is defined as a vector
which is perpendicular to both $\sv a$ and $\sv b$ with a length of $ab *
\sin\theta$, where $\theta$ is the angle between $\sv a$ and $\sv b$.  The two
vectors below both fit this definition. Both vectors have equal claim to
calling themselves the result of $\sv a \x \sv b$ 
$$
\sv a \x \sv b =
\[
\matrix{%
a_yb_z - a_zb_y \cr
a_zb_x - a_xb_z \cr
a_xb_y - a_yb_x \cr
}
\]
\mskip40mu
\sv a \x \sv b =
\[
\matrix{%
a_zb_y - a_yb_z \cr
a_xb_z - a_zb_x \cr
a_yb_x - a_xb_y \cr
}
\]
$$
The cross product must have the property that
$$
\vlen{\sv a \x \sv b} = ab * \sin\theta
$$
The cross product obeys the following identity
$$
\sv a \x \(\sv b \x \sv c\)
= \sv b \(\sv a * \sv c\) - \sv c \(\sv a * \sv b\)
$$

\paperheading{Discovery}

\noindent Let us find the area of a parallelogram spanned by the vectors $\sv
a$ and $\sv b$. We will think of $\sv a$ as the base of the parallelogram. The
parallelogram's height will then be $h = b\sin\theta$, where $\theta$ is the
angle between the $\sv a$ and $\sv b$. The parallelogram's area is
$$
A
= ab*\sin\theta
= ab*\sqrt{1 - \cos^2\theta}
= \sqrt{\(ab\)^2 - \(\sv a * \sv b\)^2}
$$
Now let us separately expand out the terms $\(ab\)^2$ and
$\(\sv a * \sv b\)^2$
$$
\eqalign{
\(ab\)^2
&=
\( \cs a_x^2 +  \cs a_y^2 +  \cs a_z^2\)
\( \cs b_x^2 +  \cs b_y^2 +  \cs b_z^2\) \cr
&=
\cs a_x^2 \cs b_x^2
+  \cs a_x^2 \cs b_y^2
+  \cs a_x^2 \cs b_z^2
+  \cs a_y^2 \cs b_x^2
+  \cs a_y^2 \cs b_y^2
+  \cs a_y^2 \cs b_z^2
+  \cs a_z^2 \cs b_x^2
+  \cs a_z^2 \cs b_y^2
+  \cs a_z^2 \cs b_z^2
\cr
\(\sv a * \sv b\)^2
&=
\(a_xb_x + a_yb_y + a_zb_z\)^2 \cr
&=
 \cs a_x^2 \cs b_x^2
+ 2a_xb_xa_yb_y
+ 2a_xb_xa_zb_z
+ \cs a_y^2 \cs b_y^2
+ 2a_yb_ya_zb_z
+ \cs a_z^2 \cs b_z^2
}
$$
Subtracting these then gives
$$
\mathcramp{0.375}
\displaylines{
\(ab\)^2 - \mskip3mu\(\sv a * \sv b\)^2
\cr
\cs a_x^2 \cs b_y^2
+  \cs a_x^2 \cs b_z^2
+  \cs a_y^2 \cs b_x^2
+  \cs a_y^2 \cs b_z^2
+  \cs a_z^2 \cs b_x^2
+  \cs a_z^2 \cs b_y^2
- 2 a_x b_x a_y b_y
- 2 a_x b_x a_z b_z
- 2 a_y b_y a_z b_z
\cr
\( \cs a_x^2 \cs b_y^2 - 2a_xb_xa_yb_y +  \cs a_y^2 \cs b_x^2\)
+ \( \cs a_x^2 \cs b_z^2 - 2a_xb_xa_zb_z +  \cs a_z^2 \cs b_x^2\)
+ \( \cs a_y^2 \cs b_z^2 - 2a_yb_ya_zb_z +  \cs a_z^2 \cs b_y^2\)
\cr
\(a_xb_y - a_yb_x\)^2
+\, \( a_x b_z - a_z b_x\)^2
+\, \( a_y b_z - a_z b_y\)^2
}
$$
The above is the square of the area, and so the area itself is
$$
A =
\sqrt{
\(a_xb_y - a_yb_x\)^2 +
\(a_x b_z - a_z b_x\)^2 +
\(a_y b_z - a_z b_y\)^2}
$$

The form of this result suggests that we have discovered some underlying
vector, and that its length is the area of the parallelogram spanned by $\sv
a$ and $\sv b$. We have the values of this vector's components at hand, but we
do not know which component should go on which axis, nor do we know the
components' signs. Notice that when $x$ and $y$ components are present $z$ is
left out, when $x$ and $z$ components are present $y$ is left out, and when
$y$ and $z$ components are present $x$ is left out. In the spirit of this
symmetry we will assign the term which mixes $x$ and $y$ to the $z$ direction,
the term which mixes $x$ and $z$ to the $y$ direction, and the term which
mixes $y$ and $z$ to the $x$ direction: We will call this new vector $\sv a \x
\sv b$:
$$
\sv a \x \sv b = \[\matrix{
\pm\(a_yb_z - a_zb_y\) \cr
\pm\(a_zb_x - a_xb_z\) \cr
\pm\(a_xb_y - a_yb_x\) \cr
}\]
$$

For the moment we do not know whether to choose $a_yb_z - a_zb_y$ or $a_zb_y -
a_yb_z$ for the $x$-axis component. There are eight possible ways of resolving
the $\pm$ signs into a concrete $+$ or $-$. But which of these eight do we
choose? None of them would break the formula for the area~$A$. However, It
turns out that there are only two ultimately correct ways to resolve the $\pm$
signs. This issue will be resolved fully in the next section.

\vskip\baselineskip

\noindent As a final note, observe that we began this whole analysis with $A =
ab*\sin\theta$. Since we calculated the area to be $A = \vlen{\sv a \x \sv b}$
we can state that $$\vlen{\sv a \x \sv b} = ab*\sin\theta$$

\paperheading{Perpendicularity}

We naturally want to know what the spatial orientation of $\sv a \x \sv b$ is
in relation to $\sv a$~and~$\sv b$. It turns out that the cross product can be
defined in a manner such that it is perpendicular to both $\sv a$ and $\sv b$.
To demonstrate this for $\sv a$ we can take a dot product
$$
\sv a * \(\sv a \x \sv b\)
=
\pm\(a_xa_yb_z - a_xa_zb_y\)
\pm\(a_ya_zb_x - a_ya_xb_z\)
\pm\(a_za_xb_y - a_za_yb_x\)
$$
Notice that this dot product comes to 0 if we choose the signs as follows
$$
\sv a * \(\sv a \x \sv b\)
=
\(a_xa_yb_z - a_xa_zb_y\)
+ \(a_ya_zb_x - a_ya_xb_z\)
+ \(a_za_xb_y - a_za_yb_x\)
= 0
$$
Yet we could also have chosen the opposite signs
$$
\sv a * \(\sv a \x \sv b\)
=
- \(a_xa_yb_z - a_xa_zb_y\)
- \(a_ya_zb_x - a_ya_xb_z\)
- \(a_za_xb_y - a_za_yb_x\)
= 0
$$

Either of these choices corresponds to a cross product that is perpendicular
to $\sv a$. By symmetry or direct computation it can be demonstrated that both
options are perpendicular to $\sv b$ as well. We therefore have
$$
\sv a \x \sv b =
\[
\matrix{%
a_yb_z - a_zb_y \cr
a_zb_x - a_xb_z \cr
a_xb_y - a_yb_x \cr
}
\]
\mskip20mu
\({\rm or}\)
\mskip20mu
\sv a \x \sv b =
\[
\matrix{%
a_zb_y - a_yb_z \cr
a_xb_z - a_zb_x \cr
a_yb_x - a_xb_y \cr
}
\]
$$
Each of these forms is the negation of the other. Both hold good to the area
formula found in the previous section, and both are perpendicular to both $\sv
a$ and $\sv b$. When working with cross products we must choose one of these
two definitions and then stick to it consistently. We will usually work with
the first of the two because it yields the elegant property $$(1, 0, 0) \x (0,
1, 0) = (0, 0, 1)$$ whereas to the second definition yields the more unusual
property $$(1, 0, 0) \x (0, 1, 0) = (0, 0, -1)$$

\paperheading{Determinant Notation}

\noindent
The cross product can be written as the pseudo determinant
$$
\sv a \x \sv b
=
\left|
\matrix{
\ihat & \jhat & \khat \cr
a_x & a_y & a_z \vrule depth 7pt width 0pt\cr
b_x & b_y & b_z \cr
}
\right|
=
\matrix{
\ihat*\(a_yb_z - a_zb_y\) \vrule depth 7pt width 0pt\cr
\jhat*\(a_zb_x - a_xb_z\) \vrule depth 7pt width 0pt\cr
\khat*\(a_xb_y - a_yb_x\) \cr
}
$$

\paperheading{Chained Cross Products}

\noindent
The chained expression $\sv a \x \(\sv b \x \sv c\)$ turns out to have a very
useful identity. To discover this identity we will first expand things out
component wise
$$
\sv a \x \(\sv b \x \sv c\)
\,=\,
\sv a\, \x
\[
\matrix{
b_y c_z - b_z c_y \cr
b_z c_x - b_x c_z \cr
b_x c_y - b_y c_x \cr
}
\]
=
\[
\matrix{
a_y\(b_x c_y - b_y c_x\) - a_z\(b_z c_x - b_x c_z\) \cr
a_z\(b_y c_z - b_z c_y\) - a_x\(b_x c_y - b_y c_x\) \cr
a_x\(b_z c_x - b_x c_z\) - a_y\(b_y c_z - b_z c_y\) \cr
}
\]
$$
This can then be rearranged to
$$
\mathcramp{0.75}
\cdots =
\[
\matrix{
a_y b_x c_y - a_y b_y c_x - a_z b_z c_x + a_z b_x c_z \cr
a_z b_y c_z - a_z b_z c_y - a_x b_x c_y + a_x b_y c_x \cr
a_x b_z c_x - a_x b_x c_z - a_y b_y c_z + a_y b_z c_y \cr
}
\]
=
\[
\matrix{
b_x\(a_yc_y + a_zc_z\) - c_x\(a_yb_y + a_zb_z\) \cr
b_y\(a_zc_z + a_xc_x\) - c_y\(a_zb_z + a_xb_x\) \cr
b_z\(a_xc_x + a_yc_y\) - c_z\(a_xb_x + a_yb_y\) \cr
}
\]
$$
We can now play a clever trick and rewrite this as
$$
\cdots =
\[
\matrix{
b_x\(\sv a * \sv c - a_xc_x\) - c_x\(\sv a * \sv b - a_xb_x\) \cr
b_y\(\sv a * \sv c - a_yc_y\) - c_y\(\sv a * \sv b - a_yb_y\) \cr
b_z\(\sv a * \sv c - a_zc_z\) - c_z\(\sv a * \sv b - a_zb_z\) \cr
}
\]
=
\[
\matrix{
b_x\(\sv a * \sv c\) - c_x\(\sv a * \sv b\) \cr
b_y\(\sv a * \sv c\) - c_y\(\sv a * \sv b\) \cr
b_z\(\sv a * \sv c\) - c_z\(\sv a * \sv b\) \cr
}
\]
$$
We have thus arrived at the identity
$$
\sv a \x \(\sv b \x \sv c\)
= \sv b \(\sv a * \sv c\)
- \sv c \(\sv a * \sv b\)
$$

\bye
