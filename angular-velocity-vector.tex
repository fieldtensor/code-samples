\input ../paper.tex

\papertitle{The Angular Velocity Vector}

\vskip\baselineskip

Let us consider a rotation about the vector $\sv\omega = (\omega_x, \omega_y,
\omega_z)$ by $\omega$ radians. Notice that a rotation about
$\ihat$ by $\omega_x$ radians, followed by a rotation about
$\jhat$ by $\omega_y$ radians, followed by a rotation about $\khat$ by
$\omega_z$ radians is not ultimately the same as a rotation about $\sv\omega$
by $\omega$ radians.

To see why these rotations are not equivalent, at least in one case, take a
any household object and rotate it $\pi/2$ radians about some $x$-axis, and
then $\pi/2$ radians about some $y$-axis. Next do this again in the reverse
order: rotate the object about the $y$-axis first and the $x$-axis second. You
will notice that the object ends up in a different orientation depending on
the order of the rotations.  Finally, notice that both of these are completely
different from a rotation about the vector $(\pi/2, \pi/2)$ by $\pi/\sqrt{2}$
radians (which is the length of that vector).

It would therefore seem that rotations do not break cleanly into components,
nor do they add together cleanly. However, it turns out that rotations
actually do both of these things under one special circumstance: namely when
the rotation angles are differentially small. The rest of this section proves
that this is indeed the case.

We know that the rotation of a vector $\sv r$ around some unit
vector $\hat\sv a$ through an angle $\phi$ can be obtained from the below
equation
$$
\sv R(\sv r, \hat\sv a, \phi)
= \hat\sv a \(\hat\sv a * \sv r \) 
+ \cos\phi\[\sv r - \hat\sv a\(\hat\sv a * \sv r \)\]
+ \sin\phi\[\hat\sv a \x \sv r\]
$$
When the angle $\phi$ is very small we will have
$$
\eqalign{
\sv R(\sv r, \hat\sv a, d\phi)
&= \(\sv r * \hat\sv a\) * \hat\sv a 
+ \sv r - \(\sv r * \hat\sv a\) * \hat\sv a
+ d\phi\[\hat\sv a \x \sv r\] \cr
&= \sv r + d\phi\[\hat\sv a \x \sv r\] \cr
}
$$
We might likewise have a rotation about some other unit vector $\hat\sv c$
through an angle $\psi$. When $\psi$ is small this will be
$$
\sv R(\sv r, \hat\sv c, d\psi)
= \sv r + d\psi\[\hat\sv c \x \sv r\]
$$
We now wish to see what happens when we composite these two small rotations
$$
\eqalign{
\sv R(\sv R(\sv r, \hat\sv a, d\phi), \hat\sv c, d\psi)
&=
\sv r + d\psi\[\hat\sv c \x \sv r\]
+ d\phi\[\hat\sv a \x \(\sv r + d\psi\[\hat\sv c \x \sv r\]\)\]
\cr
&=
\sv r + d\psi\[\hat\sv c \x \sv r\]
+ d\phi\[\hat\sv a \x \sv r\]
+ \hat\sv a \x d\phi \, d\psi\[\hat\sv c \x \sv r\]
\cr
&=
\sv r + d\psi\[\hat\sv c \x \sv r\]
+ d\phi\[\hat\sv a \x \sv r\]
\cr
}
$$
We can immediately see that a composition in the opposite order would
come out to
$$
\sv R(\sv R   (\sv r, \hat\sv c, d\psi), \hat\sv a, d\phi)
= \sv r + d\phi\[\hat\sv a \x \sv r\]
+ d\psi\[\hat\sv c \x \sv r\]
$$
This proves that composite rotations through small angles are
commutative. Next we notice that either of these results can be
rearranged to
$$
\sv r + \(\hat\sv a \, d\phi + \hat\sv c \, d\psi\) \x \sv r
$$
If we were to put another rotation into the composition it would simply add
another term to sum within the cross product. We can therefore ultimately
write
$$
d\sv r = d\sv\theta \x \sv r
$$
where $d\sv\theta$ is the sum of all the small rotation vectors, even if there
are more than just two. This result is usually divided through by $dt$ and
written as
$$
\sv v = \sv\omega \x \sv r
$$
The vector $\sv\omega$ is called the angular velocity vector, since at
any instant the vector $\sv r$ is rotating about it with angular
velocity $\omega$.

\bye
