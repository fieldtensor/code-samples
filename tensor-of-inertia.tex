\input ../paper.tex

%\papertitleND{The Tensor of Inertia}

\paperheading{Summary}

The total angular momentum of a body is
$$
\sv L = \sv R \x M\sv V + \sv I\?\sv\omega
$$
where $\sv R$ is the location of the body's center of mass, $\sv V$ is the
velocity of the body's center of mass, $\sv \omega$ is the body's angular
velocity, and $\sv I$ is the body's tensor of inertia: 
$$
\sv I =
\sum
m_i\[
\dmatrix{
\cs \rho_y^2 + \cs \rho_z^2 & -\rho_x\rho_y & -\rho_x\rho_z \cr
-\rho_y\rho_x & \cs \rho_x^2 + \cs \rho_z^2 & -\rho_y\rho_z \cr
-\rho_z\rho_x & -\rho_z\rho_y & \cs \rho_x^2 + \cs \rho_y^2 \cr
}
\]
$$
where $\sv \rho_i = (\rho_x, \rho_y, \rho_z)$ denotes the vector to the
particle $m_i$ from the center of mass. We have left out the $i$ subscripts on
the $\rho$'s here, just for brevity. Keep in mind that $\rho_x$ needs to be
read as $\rho_{ix}$, or the $x$-location of the $i$-th particle (and likewise
for $\rho_y$ and $\rho_z$).

%% \vs12pt
%% \noindent
%% The determinant of the tensor of inertia is 0, and thus it has no
%% inverse. This means that the equation \padh{0.5pt}{$\sv L =
%% I\sv\omega$} cannot be solved for $\sv\omega$ in the general case,
%% which is fascinating because it allows for a given angular momentum to
%% be expressed by a variety of different rotations.

\paperheading{Derivation}

Let us consider the angular momentum of a rigid body which is allowed to both
translate and rotate. We will denote the body's center of mass by $\sv R$. We
will say that a given particle $i$ within the body has an offset from the
center of mass of $\sv\rho_i = \sv r_i - \sv R$, where $\sv r_i$ is the
absolute location of the particle relative to a fixed origin. We will also say
that $\sv V = \sv{\dot R}$ is the velocity of the center of mass, and that $M
= \sum m_i$ is the body's mass. The body's total angular momentum is then
$$
\eqalign{
\sv L
&= \sum \sv r_i \x \sv p_i
\cr
&= \sum \sv r_i \x m_i{\dot\sv r}_i
\cr
&= \sum \(\sv \rho_i + \sv R\) \x m_i\(\sv{\dot \rho}_i + \sv V\)
\cr
&= \sum
\(\sv \rho_i \x m_i \sv{\dot\rho}_i\)
+ \(\sv \rho_i \x m_i\sv V\)
+ \(\sv R \x m_i \sv{\dot\rho}_i\)
+ \(\sv R \x m_i\sv V\)
\cr
&= \sum
\(\sv \rho_i \x m_i \sv{\dot\rho}_i\)
+ \(m_i\sv \rho_i \x \sv V\)
+ \(\sv R \x m_i \sv{\dot\rho}_i\)
+ m_i\(\sv R \x \sv V\)
\cr
&=
\sum \Bigl(\sv \rho_i \x m_i \sv{\dot\rho}_i\Bigr)
+ \(\sum m_i\sv \rho_i\) \x \sv V
+ \(\sv R \x \sum m_i \sv{\dot\rho}_i\)
+ \Bigl(\sv R \x \sv V\Bigr)\sum m_i
}
$$
Now, let us define a notation for the center of mass in a coordinate system
where the center of mass is itself the origin:
$$
\sv R' = \q{\sum m_i\sv \rho_i}{\sum m_i}
= \q{\sum m_i\sv \rho_i}{M}
$$
We then have
$$
M\sv R' = \sum m_i\sv \rho_i
$$
and also
$$
M\sv V' = \sum m_i\sv{\dot\rho}_i
$$
Our expression for the angular momentum is thus
$$
\sv L
=
\sum \Bigl(\sv \rho_i \x m_i \sv{\dot\rho}_i\Bigr)
+ \Bigl(M\sv R' \x \sv V\bigr)
+ \Bigl(\sv R \x M\sv V'\Bigr)
+ \Bigl(\sv R \x M\sv V\Bigr)
$$
But $\sv R' = 0$ and $\sv V' = 0$, since the location of the center of mass is
0 in a coordinate system where the center of mass is itself at the origin. Two
terms thus drop out of our expression, and this leave us with
$$
\sv L
=
\sv R \x M\sv V
+ \sum \sv \rho_i \x m_i \sv{\dot\rho}_i
$$
Now, a particle's distance from the center of mass $\rho_i = \vlen{\sv\rho}$
must remain constant. The velocity of the particle can thus be described with
an angular velocity vector:
$$
\sv{\dot\rho_i} = \sv\omega \x \sv\rho_i
$$
We thus have
$$
\eqalign{
\sv L
&=
\sv R \x M\sv V
+ \sum m_i\[\sv \rho_i \x \(\sv\omega \x \sv\rho_i\)\]
\cr
&=
\sv R \x M\sv V
+ \sum m_i\[\sv\omega\cs\rho_i^2 - \sv\rho_i\(\sv\rho_i * \sv\omega\)\]
}
$$
Let us consider the interesting sum $\sv\omega\cs\rho_i^2 -
\sv\rho_i\(\sv\rho_i * \sv\omega\)$. We can expand this vector expression into
component form (for now we will drop the $i$ subscripts, just for brevity):
$$
\[
\dmatrix{
\omega_x\(\cs \rho_x^2 + \cs \rho_y^2 + \cs \rho_z^2\)
- \rho_x\(\rho_x\omega_x + \rho_y\omega_y + \rho_z\omega_z\)
\vrule width 0pt height 10pt
\cr
\omega_y\(\cs \rho_x^2 + \cs \rho_y^2 + \cs \rho_z^2\)
- \rho_y\(\rho_x\omega_x + \rho_y\omega_y + \rho_z\omega_z\)
\vrule width 0pt height 12pt
\cr
\omega_z\(\cs \rho_x^2 + \cs \rho_y^2 + \cs \rho_z^2\)
- \rho_z\(\rho_x\omega_x + \rho_y\omega_y + \rho_z\omega_z\)
\vrule width 0pt height 12pt depth 6pt
}
\]
$$
When can now expand a little bit further
$$
\[
\dmatrix{
\omega_x\cs \rho_x^2 + \omega_x\cs \rho_y^2 + \omega_x\cs \rho_z^2
- \cs \rho_x^2\omega_x - \rho_x \rho_y\omega_y - \rho_x \rho_z\omega_z
\vrule width 0pt height 10pt
\cr
\omega_y\cs \rho_x^2 + \omega_y\cs \rho_y^2 + \omega_y\cs \rho_z^2
- \rho_y \rho_x\omega_x - \cs \rho_y^2\omega_y - \rho_y \rho_z\omega_z
\vrule width 0pt height 12pt
\cr
\omega_z\cs \rho_x^2 + \omega_z\cs \rho_y^2 + \omega_z\cs \rho_z^2
- \rho_z \rho_x\omega_x - \rho_z \rho_y\omega_y - \cs \rho_z^2\omega_z
\vrule width 0pt height 12pt depth 6pt
}
\]
$$
Some cancellations are possible here, and carrying them out leaves us with
$$
\[
\dmatrix{
\omega_x\cs \rho_y^2 + \omega_x\cs \rho_z^2
- \rho_x \rho_y\omega_y - \rho_x \rho_z\omega_z
\vrule width 0pt height 10pt
\cr
\omega_y\cs \rho_x^2 + \omega_y\cs \rho_z^2
- \rho_y \rho_x\omega_x - \rho_y \rho_z\omega_z
\vrule width 0pt height 12pt
\cr
\omega_z\cs \rho_x^2 + \omega_z\cs \rho_y^2
- \rho_z \rho_x\omega_x - \rho_z \rho_y\omega_y
\vrule width 0pt height 12pt depth 6pt
}
\]
$$
We can now arrange this as a linear combination of $\omega_x$, $\omega_y$, and
$\omega_z$. Note that all we are doing here is grouping terms with
parenthesis:
$$
\[
\dmatrix{
\omega_x\(\cs \rho_y^2 + \cs \rho_z^2\)
+ \omega_y\(-\rho_x \rho_y\)
+ \omega_z \(-\rho_x \rho_z\)
\vrule width 0pt height 10pt
\cr
\omega_y\(\cs \rho_x^2 + \cs \rho_z^2\)
+ \omega_x\(-\rho_y \rho_x\)
+ \omega_z\(-\rho_y \rho_z\)
\vrule width 0pt height 12pt
\cr
\omega_z\(\cs \rho_x^2 + \cs \rho_y^2\)
+ \omega_x\(-\rho_z \rho_x\)
+ \omega_y\(-\rho_z \rho_y\)
\vrule width 0pt height 12pt depth 6pt
}
\]
$$
This result now lends itself to being written in matrix form:
$$
\[
\dmatrix{
\cs \rho_y^2 + \cs \rho_z^2 & -\rho_x\rho_y & -\rho_x\rho_z \cr
-\rho_y\rho_x & \cs \rho_x^2 + \cs \rho_z^2 & -\rho_y\rho_z \cr
-\rho_z\rho_x & -\rho_z\rho_y & \cs \rho_x^2 + \cs \rho_y^2 \cr
}
\]
\[
\dmatrix{
\omega_x \cr
\omega_y \cr
\omega_z \cr
}
\]
$$
We therefore have the following (recall that you should still be visualizing
$i$ subscripts on every element of $\sv\rho$):
$$
\eqalign{
\sv L
&=
\sv R \x M\sv V
+ \sum
m_i\[
\dmatrix{
\cs \rho_y^2 + \cs \rho_z^2 & -\rho_x\rho_y & -\rho_x\rho_z \cr
-\rho_y\rho_x & \cs \rho_x^2 + \cs \rho_z^2 & -\rho_y\rho_z \cr
-\rho_z\rho_x & -\rho_z\rho_y & \cs \rho_x^2 + \cs \rho_y^2 \cr
}
\]
\[
\dmatrix{
\omega_x \cr
\omega_y \cr
\omega_z \cr
}
\]
}
$$
Every particle in the body moves (relative to the body) with the same angular
velocity vector $\sv\omega$. We can therefore factor the vector $\sv\omega$
out of the sum:
$$
\eqalign{
\sv L
&=
\sv R \x M\sv V
+ \(\sum
m_i\[
\dmatrix{
\cs \rho_y^2 + \cs \rho_z^2 & -\rho_x\rho_y & -\rho_x\rho_z \cr
-\rho_y\rho_x & \cs \rho_x^2 + \cs \rho_z^2 & -\rho_y\rho_z \cr
-\rho_z\rho_x & -\rho_z\rho_y & \cs \rho_x^2 + \cs \rho_y^2 \cr
}
\]
\)
\[
\dmatrix{
\omega_x \cr
\omega_y \cr
\omega_z \cr
}
\]
}
$$
The matrix sum in this equation is usually denoted as $\sv I$, and is called
the {\sl tensor of inertia}. The terms on the diagonal are called {\sl moments
of inertia} and those off of the diagonal are called {\sl products of
inertia}. The whole result is more simply written as
$$
\sv L = \sv R \x M\sv V + \sv I\?\sv\omega
$$

%% \paperheading{Determinant}

\bye
