\input ../paper.tex

\papertitle{Polar Vector Derivatives}

\paperheading{Summary}

\noindent
The first order derivative of a vector in polar coordinates is
$$
\dot\sv r = \dot r\sv{\hat r} + r\omega\?\sv{\hat\theta}
$$
The second order derivative of a vector in polar coordinates is
$$
\ddot\sv r
= \(\ddot r - r\?\omega^2\)\sv{\hat r}
+ \(r\alpha + 2\?\dot r\?\omega\)\sv{\hat\theta}
$$
This can also be written as
$$
\sv{\ddot r}
= \(\ddot r - r\?\omega^2\)\sv{\hat r}
+ \q1r \q d{dt}\Bigl[{r^2\omega}\Bigr]\,\sv{\hat\theta}
$$

%% This final anti-differentiation shows that the $\hat\theta$ component
%% is proportional to the torque on a body moving with the path $\sv r$.

\paperheading{Discovery of the First Derivative}

\noindent Let us consider a vector function $\sv r(t)$ at times $t_0$ and $t$.
We will use the abbreviation $\sv r = \sv r(t)$ and $\sv r_0 = \sv r(t_0)$. We
split $\sv r$ into parts $\sv\sigma$ and $\sv\gamma$ that are respectively
parallel and perpendicular to $\sv r_0$
$$
\eqalign{
\sv\sigma
&=
r\cos\theta\,\sv{\hat r_0}
\cr
\sv\gamma
&=
r\sin\theta\,\sv{\hat\theta_0}
}
$$
The vector $\sv{\hat\theta}$ here is a unit vector defined as
$$
\sv{\hat\theta} = \sv{\hat r} \x \khat
$$
The derivative of $\sv r$ is then
$$
\eqalign{
\sv{\dot r} &= \q d{dt}\mathinner{\Bigl[\sv\sigma + \sv\gamma\Bigr]}\cr
&= \dot r\cos\theta\,\sv{\hat r_0}
- r\?\omega\sin\theta\,\sv{\hat r_0}
+ \dot r\sin\theta\,\sv{\hat\theta_0}
+ r\?\omega\cos\theta\,\sv{\hat\theta_0}
\cr
}
$$
We are most interested in the derivative at time $t_0$ because at
this time we will have $\theta = 0$, which simplifies things greatly
to
$$
\dot{\sv r}
= \dot r\?\sv{\hat r} + r\?\omega\?\sv{\hat\theta}
$$
Note that we have written $\sv{\hat r}$ in place of $\sv{\hat r_0}$ and
$\sv{\hat\theta}$ in place of $\sv{\hat \theta_0}$ since at time $t = t_0$
these are the same.

\paperheading{Discovery of the Second Derivative}

\noindent One approach to finding the second derivative of $\sv r$ at time
$t_0$ would be to start from the point in the previous section before we set
$\theta = 0$. We could simply forgo setting $\theta = 0$ and instead take
another derivative with repsect to time. Afterward we could then eventually
set $\theta = 0$, and this would grant us our result. However, we will not do
this. We will instead opt for a more novel approach. Consider the following
$$
\eqalign{
\sv{\ddot r}
&= \q d{dt}\[\dot r\?\sv{\hat r} + r\?\omega\?\sv{\hat\theta}\]
\cr
&=
\ddot r\,\sv{\hat r}
+ \dot r\,\q{d\sv{\hat r}}{dt}
+ \dot r\?\omega\,\sv{\hat\theta}
+ r\?\dot\omega\,\sv{\hat\theta}
+ r\?\omega\?\q{d\sv{\hat\theta}}{dt}
}
$$
In order to proceed further we must find $d\sv{\hat r}/dt$ and
$d\sv{\hat\theta}/dt$. These are both first order vector derivatives
themselves, and so we can make use the result from the previous section.
$$
\eqalign{
\q{d\sv{\hat r}}{dt}
&=
\omega\?\sv{\hat\theta}
\cr
\q{d\sv{\hat\theta}}{dt}
&=
\q d{dt}\mathinner{\Bigl[\sv{\hat r} \x \khat\Bigr]}
= \q{d\sv{\hat r}}{dt} \x \khat
= \omega\?\sv{\hat\theta} \x \khat
= -\omega\?\sv{\hat r}
\cr
}
$$
The second derivative of $\sv r$ is then
$$
\eqalign{
\cdots
&=
\ddot r\?\sv{\hat r} + \dot r\?\omega\?\sv{\hat\theta}
+ \dot r\?\omega\?\sv{\hat\theta}
+ r\?\dot\omega\?\sv{\hat\theta}
- r\?\omega^2\?\sv{\hat r} \cr
&=
\(\ddot r - r\?\omega^2\)\sv{\hat r}
+ \(r\alpha + 2\dot r\?\omega\)\sv{\hat\theta}
}
$$
This result can also be written as
$$
\ddot \sv r =
\(\ddot r - r\?\omega^2\)\?\sv{\hat r}
+ \q1r\q d{dt}\mathinner{\Bigl[{r^2\omega}\Bigr]}\sv{\hat\theta}
$$

\bye
