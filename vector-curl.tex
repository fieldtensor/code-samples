\input ../paper.tex

\papertitle{Vector Curl}

\paperheading{Discovery (2D)}

Let us consider a vector function $\sv F(x, y)$ of two-dimensional space.
Place a square at the origin with width $\Delta x$ and height $\Delta y$. The
lower left and upper right corners will be at $(x_0, y_0)$ and $(x, y)$. Let
the square's center be at the origin. The midpoints of the square's sides will
be $x_m = 0$ and $y_m = 0$. We could write these midpoints using the literal
number $0$, but we will keep the notation $x_m$ and $y_m$ to keep the
meaning explicit.

Our goal is to find an approximate answer for the line integral of $\sv F$
over this square. We can do this by sampling the vector function at the
midpoints of the sides:
$$
\eqalign{
\oint \sv F * {\rm d}\sv c \approx
&\Bigl[\sv F(x_m, y_0) * (\Delta x, 0)\Bigr]
+ \Bigl[\sv F(x, y_m) * (0, \Delta y)\Bigr]  + \cr
&\Bigl[\sv F(x_m, y) * (-\Delta x, 0)\Bigr]
+ \Bigl[\sv F(x_0, y_m) * (0, -\Delta y)\Bigr] \cr
}
$$
We can rearrange this approximation to
$$
\eqalign{
\oint \sv F * {\rm d}\sv c \approx
&-\Bigl[
\sv F(x_m, y) 
- \sv F(x_m, y_0)
\Bigr] * (\Delta x, 0)
\cr
&
+\Bigl[
\sv F(x, y_m)
- \sv F(x_0, y_m) \Bigr] * (0, \Delta y)
\cr
}
$$
Next we can take a first order approximation for the difference terms
$$
\eqalign{
\sv F(x, y_m) - \sv F(x_0, y_m)
\approx \q{\p \sv F}{\p x}\biggl(x_0, y_m\biggr) * \Delta x
\cr
\sv F(x_m, y) - \sv F(x_m, y_0)
\approx \q{\p \sv F}{\p y}\biggl(x_m, y_0\biggr) * \Delta y
}
$$
Given the above we can now write
$$
\eqalign{
\oint \sv F * {\rm d}\sv c \approx
&
-\[ \q{\p \sv F}{\p y}\Delta y \] * (\Delta x, 0)
+\[ \q{\p \sv F}{\p x}\Delta x \] * (0, \Delta y)
\cr
\noalign{\vskip4pt}
&=
\(\q{\p F_y}{\p x} -\q{\p F_x}{\p y}\)\Delta x\?\Delta y
}
$$

This combination of partial derivatives is called the curl. For the curl to be
well defined at $(x_m, y_m)$ the functions $\p F_y/\p x$ and $\p F_x/\p y$
should be well defined at the point $(x_m, y_m)$, since this is the point
where we end up evaluating the partials as the differential square shrinks. At
the very least, the partials should approach some sensible value in the limit
as $(x, y)$ goes to $(x_m, y_m)$.

\paperheading{Discovery (3D)}

We now wish to generalize the previous analysis to three dimensions. We need
to formulate an expression for vector rotation. We want to take a vector $\sv
r$ and rotate it by angle $\theta$ about some unit axis $\sv{\hat\omega}$. You
can verify for yourself that the following expression serves exactly this
purpose:
$$
\sv R(\sv r)
=
-\?\sv{\hat\omega} \x (\sv{\hat\omega} \x \sv r) * \cos\theta
+ (\sv{\hat\omega} \x \sv r) * \sin\theta
+ \sv{\hat\omega} * \sv r
$$
This function is complicated, but fortunately we will not need to manipulate
it much in the work to come. We will only need to refer to its general
properties.

We can now use this function $\sv R$ to calculate the path integral over a
square that is rotated in some arbitrary way (and not just in the $xy$-plane).
What's more, $\sv F$ can now be a function of three inputs $\sv F(x, y, z)$:
$$
\eqalign{
\oint \sv F * {\rm d}\sv c \approx
&\Bigl[\sv F(\sv R(x_m, y_0, 0)) * \sv R(\Delta x, 0, 0)\Bigr]
+ \Bigl[\sv F(\sv R(x, y_m, 0)) * \sv R(0, \Delta y, 0)\Bigr]  + \cr
&\Bigl[\sv F(\sv R(x_m, y, 0)) * \sv R(-\Delta x, 0, 0)\Bigr]
+ \Bigl[\sv F(\sv R(x_0, y_m, 0)) * \sv R(0, -\Delta y, 0)\Bigr] \cr
}
$$
Note that $\sv R$ obeys the property
$$
\sv R(-\sv r) = -\sv R(\sv r)
$$
We can therefore rearrange the approximation just like before:
$$
\eqalign{
\oint \sv F * {\rm d}\sv c \approx
&- \Bigl[ \sv F(\sv R(x_m, y, 0)) - \sv F(\sv R(x_m, y_0, 0)) \Bigr]
* \sv R(\Delta x, 0, 0)
\cr
&+ \Bigl[ \sv F(\sv R(x, y_m, 0)) - \sv F(\sv R(x_0, y_m, 0)) \Bigr]
* \sv R(0, \Delta y, 0)
\cr
}
$$
The first order approximation of these differences are not as simple as
before. The function~$\sv F$ is now a function of three inputs as opposed to
just two, and it is now being evaluated over the function~$\sv R$ instead of
directly over Cartesian space. However, the aggregate $\sv F \circ \sv R$ is
evaluated with only $x$ or $y$ varying, while its other inputs remain fixed.
We thus have functions of just one variable in each case. Perhaps the first
order approximations are not so difficult after all. We can now rewrite our
approximation as
$$
\eqalign{
\oint \sv F * {\rm d}\sv c \approx
&-
\[
\q{\p}{\p y}\biggl[\sv F \circ \sv R\biggr]
\circ \biggl(x_m, y_0, 0\biggr)
\]
\Delta y
* \sv R(\Delta x, 0, 0)
\cr
&+ \[
\q{\p}{\p x}\biggl[\sv F \circ \sv R\biggr]
\circ \biggl(x_0, y_m, 0\biggr)
\]
\Delta x
* \sv R(0, \Delta y, 0)
\cr
}
$$
The partial derivatives in this expression come out to
$$
\displaylines{
\q{\p}{\p x}\biggl[\sv F \circ \sv R\biggr]
=
\(\q{\p \sv F}{\p x} \circ \sv R\)
\q{\p R_x}{\p x}
+
\(\q{\p \sv F}{\p y} \circ \sv R\)
\q{\p R_y}{\p x}
+
\(\q{\p \sv F}{\p z} \circ \sv R\)
\q{\p R_z}{\p x}
\cr
\noalign{\vskip6pt minus 6pt}
\q{\p}{\p y}\biggl[\sv F \circ \sv R\biggr]
=
\(\q{\p \sv F}{\p x} \circ \sv R\)
\q{\p R_x}{\p y}
+
\(\q{\p \sv F}{\p y} \circ \sv R\)
\q{\p R_y}{\p y}
+
\(\q{\p \sv F}{\p z} \circ \sv R\)
\q{\p R_z}{\p y}
}
$$
Let us now look at the partial derivatives of $\sv R$ that appear in this
result. For example, consider the derivative $\p R_y/\p x$:
$$
\eqalign{
\q{\p R_y}{\p x}
=
\left.
\q{\p}{\p x}
\middle[R_y(\sv r)\right]
=
\left.
\q{\p}{\p x}
\middle[\sv R(\sv r) * \jhat\right]
=
\sv R\(\q{\p \sv r}{\p x}\) * \jhat
=
\sv R(\ihat) * \jhat
}
$$
This is just $\jhat$ dotted with the rotation of $\ihat$ about
$\sv{\hat\omega}$. In summary we have:
$$
\q{\p R_y}{\p x} = \sv R(\?\ihat\?) * \jhat
$$
We can work out the other partial derivatives on $\sv R$ in a similar way to
get
$$
\displaylines{
\q{\p R_x}{\p x} = \sv R(\?\ihat\?) * \ihat
\hskip20pt
\q{\p R_y}{\p x} = \sv R(\?\ihat\?) * \jhat
\hskip20pt
\q{\p R_z}{\p x} = \sv R(\?\ihat\?) * \khat
\cr
\noalign{\vskip6pt minus 6pt}
\noalign{\vskip4pt}
\q{\p R_x}{\p y} = \sv R(\?\jhat\?) * \ihat
\hskip20pt
\q{\p R_y}{\p y} = \sv R(\?\jhat\?) * \jhat
\hskip20pt
\q{\p R_z}{\p y} = \sv R(\?\jhat\?) * \khat
}
$$
The partial derivatives thus reduce out
$$
\eqalign{
\left.\q{\p}{\p x}\middle[\sv F \circ \sv R\right]
&=
\left(\q{\p \sv F}{\p x} \circ \sv R\middle)
\middle(\sv R(\?\ihat\?) * \ihat\right)
+
\left(\q{\p \sv F}{\p y} \circ \sv R\middle)
\middle(\sv R(\?\ihat\?) * \jhat\right)
+
\left(\q{\p \sv F}{\p z} \circ \sv R\middle)
\middle(\sv R(\?\ihat\?) * \khat\right)
\cr
\noalign{\vskip6pt minus 6pt}
\left.\q{\p}{\p y}\middle[\sv F \circ \sv R\right]
&=
\left(\q{\p \sv F}{\p x} \circ \sv R\middle)
\middle(\sv R(\?\jhat\?) * \ihat\right)
+
\left(\q{\p \sv F}{\p y} \circ \sv R\middle)
\middle(\sv R(\?\jhat\?) * \jhat\right)
+
\left(\q{\p \sv F}{\p z} \circ \sv R\middle)
\middle(\sv R(\?\jhat\?) * \khat\right)
}
$$
For brevity let us omit writing out the nested calls to $\sv R$. If we are
being strict about our notation we should leave it in, but doing that would
coming algebra would quickly become intractable. From here onward you should
simply remember that each partial derivative of $\sv F$ is to be evaluated
over $\sv R$. Our notation will now be:
$$
\eqalign{
\left.\q{\p}{\p x}\middle[\sv F \circ \sv R\right]
&=
\left(\sv R(\?\ihat\?) * \ihat\middle)
\q{\p \sv F}{\p x}\right.
+
\left(\sv R(\?\ihat\?) * \jhat\middle)
\q{\p \sv F}{\p y}\right.
+
\left(\sv R(\?\ihat\?) * \khat\middle)
\q{\p \sv F}{\p z}\right.
\cr
\noalign{\vskip6pt minus 6pt}
\left.\q{\p}{\p y}\middle[\sv F \circ \sv R\right]
&=
\left(\sv R(\?\jhat\?) * \ihat\middle)
\q{\p \sv F}{\p x}\right.
+
\left(\sv R(\?\jhat\?) * \jhat\middle)
\q{\p \sv F}{\p y}\right.
+
\left(\sv R(\?\jhat\?) * \khat\middle)
\q{\p \sv F}{\p z}\right.
}
$$
We can now write our approximation as
$$
\eqalign{
\oint \sv F * {\rm d}\sv c
\approx
&-\[
\left(\sv R(\?\jhat\?) * \ihat\middle)
\q{\p \sv F}{\p x}\right.
+
\left(\sv R(\?\jhat\?) * \jhat\middle)
\q{\p \sv F}{\p y}\right.
+
\left(\sv R(\?\jhat\?) * \khat\middle)
\q{\p \sv F}{\p z}\right.
\]
\Delta y * \sv R(\Delta x, 0, 0)
\cr
&+
\[
\left(\sv R(\?\ihat\?) * \ihat\middle)
\q{\p \sv F}{\p x}\right.
+
\left(\sv R(\?\ihat\?) * \jhat\middle)
\q{\p \sv F}{\p y}\right.
+
\left(\sv R(\?\ihat\?) * \khat\middle)
\q{\p \sv F}{\p z}\right.
\]
\Delta x * \sv R(0, \Delta y, 0)
}
$$
Next notice that $\sv R$ obeys
$$
\sv R(c\?\sv r) = c\?\sv R(\sv r)
$$
and thus that
$$
\eqalign{
\sv R(0, \Delta y, 0)
&=
\sv R(0, 1, 0) \, \Delta y
= \sv R(\jhat) \, \Delta y
\cr
\sv R(\Delta x, 0)
&=
\sv R(1, 0, 0) \, \Delta x
=
\sv R(\ihat) \, \Delta x \cr
}
$$
The next logical iteration of our approximation is therefore
$$
\eqalign{
\oint \sv F * {\rm d}\sv c
\approx
&-\[
\left(\sv R(\?\jhat\?) * \ihat\middle)
\q{\p \sv F}{\p x}\right.
+
\left(\sv R(\?\jhat\?) * \jhat\middle)
\q{\p \sv F}{\p y}\right.
+
\left(\sv R(\?\jhat\?) * \khat\middle)
\q{\p \sv F}{\p z}\right.
\]
* \sv R(\?\ihat\?) \; \Delta y \, \Delta x
\cr
&+
\[
\left(\sv R(\?\ihat\?) * \ihat\middle)
\q{\p \sv F}{\p x}\right.
+
\left(\sv R(\?\ihat\?) * \jhat\middle)
\q{\p \sv F}{\p y}\right.
+
\left(\sv R(\?\ihat\?) * \khat\middle)
\q{\p \sv F}{\p z}\right.
\]
* \sv R(\?\jhat\?) \; \Delta x \, \Delta y
}
$$
We can now do some trivial vector algebra to pull out the partials of $\sv F$:
$$
\eqalign{
\oint \sv F * {\rm d}\sv c
\hskip2pt\approx\hskip4pt
&+
\q{\p \sv F}{\p x}
*\[
\Bigl[\sv R(\?\ihat\?) * \ihat\Bigr] \sv R(\?\jhat\?)
-
\Bigl[\sv R(\?\jhat\?) * \ihat\Bigr] \sv R(\?\ihat\?)
\]
\Delta x \, \Delta y
\cr
&+
\q{\p \sv F}{\p y}
*
\[
\Bigl[\sv R(\?\ihat\?) * \jhat\Bigr] \sv R(\?\jhat\?)
-
\Bigl[\sv R(\?\jhat\?) * \jhat\Bigr] \sv R(\?\ihat\?)
\]
\Delta x \, \Delta y
\cr
&+
\q{\p \sv F}{\p z}
*
\[
\Bigl[\sv R(\?\ihat\?) * \khat\Bigr] \sv R(\?\jhat\?)
-
\Bigl[\sv R(\?\jhat\?) * \khat\Bigr] \sv R(\?\ihat\?)
\]
\Delta x \, \Delta y
\cr
}
$$
The inner terms involving the $\sv R$'s and the basis vectors are really just
triple cross products. According to $\sv A \x (\sv B \x \sv C) = \sv B(\sv A *
\sv C) - \sv C(\sv A * \sv B)$ we can reduce things down to
$$
\eqalign{
\oint \sv F * {\rm d}\sv c
\hskip2pt\approx\hskip4pt
&+
\left.
\q{\p \sv F}{\p x}
*\middle[
\hskip1pt \sv{\hat\imath} \hskip2pt
\x \middle(\sv R(\?\jhat\?) \x \sv R(\?\ihat\?)\middle)
\right]
\Delta x \, \Delta y
\cr
&+
\left.
\q{\p \sv F}{\p y}
*\middle[
\hskip1pt \sv{\hat\jmath} \hskip2pt
\x \middle(\sv R(\?\jhat\?) \x \sv R(\?\ihat\?)\middle)
\right]
\Delta x \, \Delta y
\cr
&+
\left.
\q{\p \sv F}{\p z}
*\middle[
\hskip1pt \sv{\hat k} \hskip2pt
\x \middle(\sv R(\?\jhat\?) \x \sv R(\?\ihat\?)\middle)
\right]
\Delta x \, \Delta y
\cr
}
$$
The cross product $\sv R(\jhat) \x \sv R(\ihat)$ is just a cross of the second
and first basis the vectors of our rotated coordinate system, namely $\sv
R(\jhat) \x \sv R(\ihat) = \sv R(\sv{\hat k})$. It must
therefore be the third basis vector of that rotated system, but negated. When
we take this into account we get
$$
\eqalign{
\oint \sv F * {\rm d}\sv c
\approx
\left.
\q{\p \sv F}{\p x}
*\middle[
\sv{\hat\imath}
\x \sv R(\sv{\hat k})
\right]
\Delta x \, \Delta y
+
\left.
\q{\p \sv F}{\p y}
*\middle[
\sv{\hat\jmath}
\x \sv R(\sv{\hat k})
\right]
\Delta x \, \Delta y
+
\left.
\q{\p \sv F}{\p z}
*\middle[
\sv{\hat k}
\x \sv R(\sv{\hat k})
\right]
\Delta x \, \Delta y
\cr
}
$$
We can do more with this if we rearrange how the triple products are
organized, such that $\sv R(\sv{\hat k})$ appears as the outermost term. We can also
pull out the factors of $\Delta x \? \Delta y$:
$$
\eqalign{
\oint \sv F * {\rm d}\sv c
\approx
\(\left.
\sv R(\sv{\hat k})
*\middle[
\q{\p \sv F}{\p x}
\x \sv{\hat\imath}
\right]
+
\left.
\sv R(\sv{\hat k})
*\middle[
\q{\p \sv F}{\p y}
\x \sv{\hat\jmath}
\right]
+
\left.
\sv R(\sv{\hat k})
*\middle[
\q{\p \sv F}{\p z}
\x \sv{\hat k}
\right]
\)
\Delta x \, \Delta y
\cr
}
$$
In fact, we can now factor out the $\sv R(\sv{\hat k})$ term altogether:
$$
\eqalign{
\oint \sv F * {\rm d}\sv c
\approx
\(\left.
\middle[
\q{\p \sv F}{\p x}
\x \sv{\hat\imath}
\right]
+
\left.
\middle[
\q{\p \sv F}{\p y}
\x \sv{\hat\jmath}
\right]
+
\left.
\middle[
\q{\p \sv F}{\p z}
\x \sv{\hat k}
\right]
\)
*
\sv R(\sv{\hat k})
\,
\Delta x \, \Delta y
\cr
}
$$
When we expand out the cross products we get
$$
\oint \sv F * {\rm d}\sv c
\approx
\[
\q{\p F_y}{\p x}\khat
- \q{\p F_z}{\p x}\jhat
+ \q{\p F_z}{\p y}\ihat
- \q{\p F_x}{\p y}\khat
+ \q{\p F_x}{\p z}\jhat
- \q{\p F_y}{\p z}\ihat
\]
* \sv R(\sv\khat)
\, \Delta x \, \Delta y
$$
Next we can group the terms around their basis vectors:
$$
\oint \sv F * {\rm d}\sv c
\approx
\[
\(\q{\p F_z}{\p y}
- \q{\p F_y}{\p z}\)\ihat \:
+ \(\q{\p F_x}{\p z}
- \q{\p F_z}{\p x}\)\jhat \:
+ \(\q{\p F_y}{\p x}
- \q{\p F_x}{\p y}\)\khat \,
\]
* \sv R(\sv\khat)
\, \Delta x \, \Delta y
$$
When we take the square to be differentially small the result goes from
approximate to exact and we write $\Delta x \, \Delta y$ as the differential
area $dx\,dy$:
$$
\oint \sv F * {\rm d}\sv c
=
\[
\(\q{\p F_z}{\p y}
- \q{\p F_y}{\p z}\)\ihat \:
+ \(\q{\p F_x}{\p z}
- \q{\p F_z}{\p x}\)\jhat \:
+ \(\q{\p F_y}{\p x}
- \q{\p F_x}{\p y}\)\khat \,
\]
* \sv R(\sv\khat)
\, dx \, dy
$$
For brevity let us define a notation for this new vector quantity
$$
\nabla \x \sv F = 
\(\q{\p F_z}{\p y}
- \q{\p F_y}{\p z}\)\ihat \:
+ \(\q{\p F_x}{\p z}
- \q{\p F_z}{\p x}\)\jhat \:
+ \(\q{\p F_y}{\p x}
- \q{\p F_x}{\p y}\)\khat \,
$$
The vector $\sv R(\khat)\, dx \, dy$ is normal to the surface of the square,
and it has a magnitude equal to the square's area. We can denote it more
concisely as $d^2\!\?\sv A$:
$$
d^2\!\?\sv A = \sv R(\khat)\,\Delta x\Delta y
$$
Our path integral can now be written as
$$
\oint \sv F * {\rm d}\sv c
=
\(\nabla \x \sv F\) * d^2\!\,\sv A
$$
After much work we finally have our answer. It appears that taking a path
integral of $\sv F$ around a very small square is equivalent to taking the dot
product of that square's are vector with this new vector $\nabla \x \sv F$,
which is composed of various partials of $\sv F$. Since the path integral
depends on $\sv R(\khat)$ the result will differ when taken over two squares
which are centered at the same point but oriented differently. However, the
vector $\nabla \x \sv F$ is clearly independent of the orientation of the
square around which the path integral is being taken. The path integral's
value will be maximized when $d^2{\sv A}$ points in the direction of $\nabla
\x \sv F$.

\bye
