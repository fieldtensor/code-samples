\input ../paper.tex

\papertitle{The Properties of a Rigid Body}

\paperheading{Summary}

\noindent
A rigid body's angular momentum is
$$
\sv L = \sv R \x M\sv V + \sv I\?\sv\omega
$$
A rigid body's kinetic energy is
$$
{\rm KE} = \q12 MV^2 + \q12 \(\sv I\?\sv\omega * \sv\omega\)
$$
If a small departure from the usual order of operations can be
forgiven then this can be written more mnemonically as
$$
{\rm KE} = \q12 MV^2 + \q12 \sv I\?\sv\omega^2
$$

\paperheading{Angular Momentum}

\noindent The angular momentum held by a rigid body can be split into
translational and rotational components. Let us denote the body's center of
mass as $\sv R$ and the velocity of the center of mass as $\sv V$. We will
express the position of a given particle in the body as an offset $\sv\rho_i$
from the body's center of mass. We can then express $\sv r_i$ and $\sv v_i =
\sv{\dot r}_i$ in terms of $\sv\rho_i$ and $\sv V\!\? = \dot\sv R$
$$
\eqalign{
\sv r_i &= \sv R + \sv\rho_i \cr
\sv v_i &= \sv V + \sv{\dot\rho}_i
}
$$
The body's angular momentum then comes out to
$$
\eqalign{
\sv L
&=
\sum \sv r_i \x m_i\sv v_i
\cr
&=
\sum
\[
\mathinner{\Bigl(\sv R + \sv \rho_i\Bigr)}
\x m_i\mathinner{\Bigl(\sv V + \sv{\dot\rho}_i\Bigr)}
\]
\cr
&=
\sum
\mathinner{\Bigl[
\sv R \x m_i\sv V
+ \sv R \x m_i\sv{\dot\rho}_i
+ m_i\sv\rho_i \x \sv V
+ \sv\rho_i \x m_i\sv{\dot\rho}_i
\Bigr]}
\cr
&=
\sv R \x M\sv V
+ \sv R \x \(\sum m_i\sv{\dot\rho}_i\)
+ \(\sum m_i\sv\rho_i\) \x \sv V
+ \sum\mathinner{\Bigl(\sv\rho_i \x m_i\sv{\dot\rho}_i\Bigr)}
\cr
&=
\sv R \x M\sv V
+ \sum\mathinner{\Bigl(\sv\rho_i \x m_i\sv{\dot\rho}_i\Bigr)}
\cr
&=
\sv R \x M\sv V
+ \sum \sv I_i\?\sv\omega \cr
&=
\sv R \x M\sv V
+ \sv I\?\sv\omega
}
$$
Note that in the fourth line we relied on certain sums coming out to $\sv 0$,
specifically  $\sum m_i\sv{\dot\rho}_i = \sv 0$ and $\sum m_i\sv\rho_i = \sv
0$. We know this to be true because the sums express the position and velocity
of the center of mass in the very coordinate system of the center of mass.
The symbol $\sv I_i$ denotes the tensor of inertia about the center of mass of
a given particle, and $\sv I$ is then the sum of those tensors. This brings us
to our next section, which is very brief.

\paperheading{The Aggregate Tensor of Inertia}

\noindent
A rigid body has a tensor of inertia $\sv I$ defined as
$$
\sv I = \sum \sv I_i
$$
This tensor plays a key role is expressing both the body's angular momentum
and energy. Note that the tensor is dependent not just on the body's shape,
but also on its orientation in space. For this reason the tensor of inertia is
usually time dependent. As the body rotates, the tensor changes.

\paperheading{Kinetic Energy}

\noindent The kinetic energy held by a rigid body can be split into
translational and rotational components. To make this split we will make use
of the $\sv\rho_i$ notation from the previous section on angular momentum.
Also as before, we will drop any terms which express the position or velocity
of the center of mass in the very coordinate system of the center of mass,
because such terms evaluate to $\sv 0$ by definition.
$$
\eqalign{
{\rm KE}
&=
\sum \q12\? m_i \cs v_i^2
\cr
&=
\sum \q12\? m_i \mathinner{\Bigl(\sv V + \sv{\dot\rho}_i\Bigr)^2}
\cr
&=
\sum \q12\? m_i \sv V^2
+ m_i\mathinner{\Bigl(\sv V * \sv{\dot\rho}_i\Bigr)}
+ \q12\? m_i \cs \sv{\dot\rho}_i^2
\cr
&=
\q12 MV^2
+ \sv V * \sum m_i \sv{\dot\rho}_i
+ \q12 \sum m_i\mathinner{\bigl(\sv \omega \x \sv \rho_i\bigr)}^2
\cr
&=
\q12 MV^2
+
\q12
\sum
m_i
\mathinner{
\Bigl(
\sv \omega * \mathinner{\Bigl(\sv\rho_i \x \mathinner{\Bigl(\sv \omega \x \sv\rho_i\Bigr)}\Bigr)}
\Bigr)
}
\cr
&=
\q12 MV^2
+ \q12\? \sv\omega * \sum\Bigl(
\sv\rho_i \x m_i\sv{\dot\rho}_i\Bigr)
\cr
&=
\q12 MV^2 + \q12 \mathinner{\Bigl(\sv\omega * \sv L\Bigr)}
\cr
}
$$
The move from the fourth line to the fifth line here might seem mysterious.
If so, recall the vector identity $\sv A * \(\sv B \x \sv C\) = \sv C * \(\sv
A \x \sv B\)$, and with this in mind it should make more sense.

\vskip\baselineskip

\noindent If a small departure from the usual order of operations can be
forgiven then this can be written as
$$
{\rm KE} = \q12 MV^2 + \q12 \sv I\?\sv\omega^2
$$
Be careful to remember that in this mnemonic form the term $\sv
I\?\sv\omega^2$ should not be read as $\sv I\vlen{\sv\omega}^2$, but rather as
$\sv I\?\sv\omega * \sv\omega$

\bye
