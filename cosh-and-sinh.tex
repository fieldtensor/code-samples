\input ../paper.tex

\papertitle{The Hyperbolic Functions}

\paperheading{Summary}

The hyperbolic functions take hyperbolic areas as inputs, and they return the
locations on the hyperbola which produce those areas. They are
$$
\eqalign{
\cosh(x) &= \q{e^x + e^{-x}}{2} \cr
\sinh(x) &= \q{e^x - e^{-x}}{2} \cr
}
$$
These functions obey the identity
$$
\cosh{}^2 - \sinh{}^2 = 1
$$
Additionally, they are each others derivatives:
$$
\eqalign{
\cosh{}' &= \sinh{} \cr
\sinh{}' &= \cosh{} \cr
}
$$
It is not difficult to show by manually expanding out the exponentials that
$$
\eqalign{
\cosh(x + y) &= \cosh(x)\cosh(y) + \sinh(x)\sinh(y) \cr
\sinh(x + y) &= \cosh(x)\sinh(y) + \cosh(y)\sinh(x) \cr
}
$$
The inverses of these functions are
$$
\eqalign{
\arccosh(x) &= \log\(x \pm \sqrt{x^2 - 1}\;\) \cr
\arcsinh(x) &= \log\(x - \sqrt{1 + x^2}\;\) \cr
}
$$
Note that either $+$ or $-$ is allowed inside of the $\log$ for $\arccosh$,
but not for $\arcsinh$.

\paperheading{Discovery}

Consider a function $f(x)$ which gives the unit hyperbola:
$$
f(x) = \sqrt{x^2 - 1}
$$
Now, let us consider the area
$$
A(x)
= xf - 2\int_1^x f(x) \; dx
= \(x\sqrt{x^2 - 1}\;\) + 2\int_1^x \sqrt{x^2 - 1} \; dx
$$
The derivative of this is
$$
A'(x) = \[\sqrt{x^2 - 1} + \q{x^2}{\sqrt{x^2 - 1}}\] - 2\sqrt{x^2 - 1}
= \q{x^2 - 1 + x^2 - 2\(x^2 - 1\)}{\sqrt{x^2 - 1}}
= \q{1}{\sqrt{x^2 - 1}}
$$
Now, let us call the inverse of $A$ by the name $f(x)$. We know that $f(x)$
must obey
$$
\q{d}{dx}\biggr[A(f(x))\biggl]
= A'(f(x)) * f'(x)
= \q{1}{\sqrt{f(x)^2 - 1}} * f'(x) = 1
$$
or in other words
$$
f' = \sqrt{f^2 - 1}
$$
Let us square this:
$$
{f'}^2 = f^2 - 1
$$
and then take another derivative:
$$
2f'f'' = 2ff'
$$
We now have
$$
f'' = f
$$
We know from any basic study of differential equations that the solution to
this is
$$
f(x) = \alpha e^x + \beta e^{-x}
$$
for some values $\alpha$ and $\beta$. Now, we know that the area function will
have $A(1) = 0$, and so we have $f(0) = 1$. Moreover, we know from $f' =
\sqrt{f^2 - 1}$ that $f'(0) = 0$, since $f(0) = 1$. Based on this we know that
$$
\eqalign{
1 &= \alpha + \beta \cr
0 &= \alpha - \beta
}
$$
and so $\alpha$ and $\beta$ are both $1/2$. The inverse function of $A$ is
therefore
$$
f(x) = \q{e^x + e^{-x}}{2}
$$
This function $f$ is usually called $\cosh$:
$$
\cosh(x) = \q{e^x + e^{-x}}{2}
$$
The $\cosh$ function takes a hyperbolic area as an input, and it identifies
which $x$-axis location on the parabola will create the given area. Since
$\cosh$ is a horizontal location on the parabola we can expect that there be
some other function $\sinh$ which obeys
$$
\cosh^2 - \sinh^2 = 1
$$
From this we can derive an expression for $\sinh$:
$$
\sinh
= \sqrt{\cosh^2 - 1}
= \sqrt{\q{e^{2x} + 2 + e^{-2x}}{4} - 1}
= \sqrt{\q{e^{2x} - 2 + e^{-2x}}{4}}
= \q{e^{x} - e^{-x}}{2}
$$

\paperheading{Derivatives}

It is immediately apparent that $\cosh$ and $\sinh$ are derivatives of each
other:
$$
\eqalign{
\cosh'
&= \q{d}{dx}\biggl[\q{e^x + e^{-x}}{2}\biggr]
= \q{e^x - e^{-x}}{2}
= \sinh
\cr
\sinh'
&= \q{d}{dx}\biggl[\q{e^x - e^{-x}}{2}\biggr]
= \q{e^x + e^{-x}}{2}
= \cosh
}
$$

\paperheading{Inverses}

The inverse of $\cosh$ is not difficult to find
$$
\eqalign{
\cosh{} &= \q{e^x + e^{-x}}{2} \cr
2\cosh{} &= e^x + e^{-x} \cr
e^x2\cosh{} &= {e^x}^2 + 1 \cr
0 &= {e^x}^2 - e^x2\cosh{} + 1 \cr
e^x &= \q{2\cosh{}\pm\sqrt{4\cosh{}^2 - 4\;}}{2} \cr
e^x &= \cosh{}\pm\sqrt{\cosh{}^2 - 1} \cr
x &= \log\(\cosh{}\pm\sqrt{\cosh{}^2 - 1}\;\)
}
$$
The inverse of $\cosh$ is therefore
$$
\arccosh(x) = \log\(x \pm \sqrt{x^2 - 1}\;\)
$$
Typically the $\pm$ is resolved to $+$, but $-$ is allowable as well. The
inverse of $\sinh$ is easy to find as well
$$
\eqalign{
\sinh{} &= \q{e^x - e^{-x}}{2} \cr
2\sinh{} &= e^x - e^{-x} \cr
e^x2\sinh{} &= {e^x}^2 - 1 \cr
0 &= {e^x}^2 - e^x2\sinh{} - 1 \cr
e^x &= \q{2\sinh{}\pm\sqrt{4\sinh{}^2 + 4\;}}{2} \cr
e^x &= \sinh{}\pm\sqrt{\sinh{}^2 + 1} \cr
x &= \log\(\sinh{}\pm\sqrt{\sinh{}^2 + 1}\;\)
}
$$
The inverse of $\sinh$ is therefore
$$
\arcsinh(x) = \log\(x + \sqrt{1 + x^2}\;\)
$$
Note that we must resolve the $\pm$ to $+$, unlike with $\arccosh$ where $-$
is allowed as well. Resolving $\pm$ to $-$ for $\arcsinh$ would mean taking
the log of a negative value.

\paperheading{An Oddity of arccosh}

Recall that $\arccosh$ is
$$
\arccosh(x) = \log\(x \pm \sqrt{x^2 - 1}\;\)
$$
and that its derivative is
$$
\arccosh'(x) = \q{1}{\sqrt{x^2 - 1}}
$$
Now, as $x$ goes to $\infty$ we see that $\arccosh$ approaches
$$
\arccosh(x) \approx \log\(2x\)
$$
and that its derivative approches 
$$
\arccosh'(x) \approx \q{1}{x}
$$
How can this be? Should it not be the case that the derivative approaches
$1/2x$, since the function itself becomes $\log(2x)$? So far, this remains a
mystery to me.

\bye
