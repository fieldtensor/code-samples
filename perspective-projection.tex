\input ../paper.tex

\prot\def\matBigStrut{\vrule width 0pt height 11.5pt depth 6.5pt}

\papertitle{Perspective Projection}

\paperheading{Summary}

The projection matrix is
$$
\left[\dmatrix{
\matBigStrut
\cot\(\q{\gamma}{2}\)\big/\,\sigma & 0 & 0 & 0 \cr
\matBigStrut
0 & \cot\(\q{\gamma}{2}\) & 0 & 0 \cr
0 & 0 &
\q{\alpha + \beta}{\alpha - \beta} &
\matBigStrut
\q{2\alpha\beta}{\alpha - \beta} \cr
\matBigStrut
0 & 0 & -1 & 0 \cr
}\right]
$$
In this matrix $\gamma$ is the vertical field of view angle, $\sigma$ is the
width-to-heigh ratio, and $\alpha$ and $\beta$ are the distances of the
clipping planes from the origin. The values of $\alpha$ and $\beta$ are always
positive, though the clipping planes themselves sit on the negative $z$-axis.
For this matrix to work the incoming vector must have its fourth component set
to $1$.

\paperheading{Discovery}

In a typical graphics system all geometry must end up inside of the normalized
space
$$
{\rm NDC} = [-1, 1] \x [-1, 1] \x [-1, 1]
$$
The acronym NDC is short for ``normalized device coordinates.'' The range
$[-1, 1]$ on the $x$-axis is mapped to the viewport's horizontal pixel space,
and $[-1, 1]$ on the $y$-axis is mapped to the vertical pixel space. The range
$[-1, 1]$ on the $z$ axis is mapped to the depth buffer.

We need to find a method of projecting all visible geometry into this small
NDC cube, and in a perspective-correct way. Let us say that the eye sits at
the origin and looks down the negative $z$-axis. We will place the near
clipping plane down the negative $z$-axis at a distance of $\alpha$ from the
origin, and the far clipping plane at a distance of $\beta$. Note that the
clipping planes are located on the negative $z$-axis, but the values $\alpha$
and $\beta$ are always taken to be positive.

A given point $\sv Q$ has a vector that connects it to the eye. We need to
intersect that vector with the near and far clipping planes. We will consider
the near clipping plane first. Let $\sv A$ be the point at which $\sv Q$
intersects the near clipping plane. It will then have to be the case that
$$
\eqalign{
A_z &= -\alpha \cr
\sv Q \x \sv A &= 0 \cr
}
$$
The first equation expresses that $\sv A$ lies on the near clipping plane, and
the second expresses that $\sv A$ is collinear with $\sv Q$. If we expand out
the second equation we get
$$
\eqalign{
Q_yA_z - Q_zA_y &= 0 \cr
Q_zA_x - Q_xA_z &= 0 \cr
Q_xA_y - Q_yA_z &= 0 \cr
}
$$ 
But we already know that $A_z = -\alpha$, so let us rewrite the above with
this taken into account:
$$
\eqalign{ -Q_y\alpha - Q_zA_y &= 0 \cr Q_zA_x + Q_x\alpha &= 0 \cr Q_xA_y -
Q_yA_x &= 0 \cr } $$ We can now rearrange these to solve for the other
components of $\sv A$. We will drop the third equation, because we can solve
for $\sv A$ with just the first two:
$$
\eqalign{
A_y &= -\alpha * Q_y/ Q_z \cr
A_x &= -\alpha * Q_x / Q_z \cr
}
$$

Recall that we have to map these $x$ and $y$ values into the range $[-1, 1]$.
Let's start with $A_y$. Think of the near clipping plane as being cut into two
pieces, one above the $xz$-plane and one below. If $\gamma$ is the field of
view angle then the height $h$ of either half can be computed by treating the
value $\tan(\gamma / 2)$ as a slope, and taking
$$
h = \alpha * \tan(\gamma / 2)
$$
We need only divide $A_y$ by this value in order to map it into the range
$[-1, 1]$. We will call the mapped result $N_y$ ($N$ for NDC):
$$
N_y
= A_y / h
= \q{-\alpha * Q_y / Q_z}{\alpha * \tan(\gamma / 2)}
= -\q{Q_y * \cot(\gamma / 2)}{Q_z}
$$
Unless the viewport is a square $A_x$ will have to be divided by a different
value. Let us give the name $\sigma$ to the width-to-height ratio of the
viewport. The relevant width will then be
$$
w = \sigma h
$$
The relevant $N_x$ value will then be
$$
N_x
= A_x / w
= \q{-\alpha * Q_y / Q_z}{\sigma * \alpha * \tan(\gamma / 2)}
= -\q{Q_x * \cot(\gamma / 2)}{\sigma * Q_z}
$$

At this point we can't help but notice that this relation isn't linear, and so
it can't be represented as a matrix. The division by $-1/Q_x$ is the key
problem. We fix this non-linearity in the following way. Let us define a new
intermediate coordinate system. We will say the transformation of $\sv Q$ into
this coordinate system is called $\sv C$, where $\sv C$ is given by the
following transformation (ignore the unknown elements for now):
$$
\sv C
= \left[\dmatrix{
\matBigStrut
\cot\(\q{\gamma}{2}\)\big/\,\sigma & 0 & 0 & 0 \cr
\matBigStrut
0 & \cot\(\q{\gamma}{2}\) & 0 & 0 \cr
\matBigStrut
? & ? & ? & ? \cr
\matBigStrut
0 & 0 & -1 & 0 \cr
}\middle]
\middle[\matrix{
\matBigStrut
Q_x \cr
\matBigStrut
Q_y \cr
\matBigStrut
Q_z \cr
\matBigStrut
?
}\right]
$$
We will then say that the final point in NDC space $\sv N$ is given by
$$
\sv N = \sv C / C_w
$$
Notice, of course, that the matrix we used ensures that $C_w = -Q_z$. This
forced non-linear operation is called ``perspective division,'' and it is
widely supported across many graphics systems.

Next let us focus on calculating $N_z$. We have to find some expression for
$C_z$ which will divide sensibly by $-Q_z$ to yield a depth value in the range
$[-1, 1]$. On the extreme ends we want to obtain $-1$ when $Q_z = -\alpha$ and
$+1$ when $Q_z = -\beta$. Thus $C_z$ must be in the range $[-\alpha,
\beta\?]$. Observe that our point $\sv Q$ can be expressed as
$$
\sv Q = \sv A + d * (\sv B - \sv A)
$$
where $d$ is some value between $0$ and $1$. Let us consider this the
$z$-components of this equation so that we can solve for $d$:
$$
\eqalign{
Q_z &= A_z + d * (B_z - A_z) \cr
Q_z &= -\alpha + d * (-\beta + \alpha) \cr
Q_z + \alpha &= d * (-\beta + \alpha) \cr
d &= \q{Q_z + \alpha}{\alpha - \beta} \cr
}
$$
Since we desire that $C_z$ be in the range $[-\alpha, \beta\?]$ we can set
$$
\eqalign{
C_z &= -\alpha + d\(\alpha + \beta\) \cr
&= -\alpha + \(\alpha + \beta\) * \q{Q_z + \alpha}{\alpha - \beta} \cr
&=
\q
{-\alpha\(\alpha - \beta\) + \(\alpha + \beta\)\(Q_z + \alpha\)}
{\alpha - \beta}
\cr
&=
\q
{-\alpha^2 + \alpha\beta + Q_z\(\alpha + \beta\) + \alpha\beta + \alpha^2}
{\alpha - \beta}
\cr
&=
\q
{\alpha\beta + Q_z\(\alpha + \beta\) + \alpha\beta}
{\alpha - \beta}
\cr
&=
Q_z
\q
{\alpha + \beta}
{\alpha - \beta}
+
\q
{2\alpha\beta}
{\alpha - \beta}
}
$$
Here we have again derived an expression that doesn't look like a typical
matrix-vector multiplication. In a general matrix-vector multiplication every
term will have one element from the matrix and one element from the vector.
Yet here we have this stray term~$2\alpha\beta/(\alpha - \beta)$. Fortunately
we can fit this term into the matrix multiplication if we require that $Q_w =
1$. Our final result is thus
$$
\sv C
= \left[\dmatrix{
\matBigStrut
\cot\(\q{\gamma}{2}\)\big/\,\sigma & 0 & 0 & 0 \cr
\matBigStrut
0 & \cot\(\q{\gamma}{2}\) & 0 & 0 \cr
0 & 0 &
\q{\alpha + \beta}{\alpha - \beta} &
\matBigStrut
\q{2\alpha\beta}{\alpha - \beta} \cr
\matBigStrut
0 & 0 & -1 & 0 \cr
}\middle]
\middle[\matrix{
\matBigStrut
Q_x \cr
\matBigStrut
Q_y \cr
\matBigStrut
Q_z \cr
\matBigStrut
1
}\right]
$$

There is one aspect of this matrix which we have so far neglected to mention,
which is important in some graphics systems. Recall that the expression for
$A_x$ was
$$
A_x = -\alpha * Q_x / Q_z
$$
As we continued on from this we chose to construct the matrix such that we
would have
$$
C_w = -Q_z
$$
We could have just as easily set flipped the $-1$ on the last row to a $1$, in
which case we would have
$$
C_w = Q_z
$$
In this case we would have to negate terms like $\cot(\gamma/2)/\sigma$ to get
the negative sign back into the expression for $A_x$. Now, recall that visible
points will have negative values for $Q_z$, since the eye looks down the
negative $z$-axis. This will mean that the choice \hbox{$C_w = -Q_z$} will
make $C_w$ positive. This is important because OpenGL will automatically clip
away points for which any of following inequalities fail:
$$ 
\displaylines{
-C_w \le C_x \le C_w \cr
-C_w \le C_y \le C_w \cr
-C_w \le C_z \le C_w \cr
}
$$
These inequalities would be false by definition if $C_w$ were to be assigned a
negative value. This is exactly why we chose to have $-1$ on the fourth row of
our matrix, as opposed to~$1$.

\bye
