\input ../paper.tex

\papertitle{Fourier Series}

\paperheading{Summary}

Any periodic function with period $\tau$ can be written as
$$
f(t) =
\sum_{n = 0}^{\infty} 
\alpha_n \cos(\omega_n t) + \beta_n \sin(\omega_n t)
$$
where the coefficient are determined by
$$
\eqalign{
\alpha_n 
&=
\q{2}{\tau}
\int_{-\tau/2}^{\tau/2}
f(t)*\cos\(\q{2\pi}{\tau}n * t\) \, dt
\cr
\noalign{\vskip6pt}
\beta_n 
&=
\q{2}{\tau}
\int_{-\tau/2}^{\tau/2}
f(t)*\sin\(\q{2\pi}{\tau}n * t\) \, dt
}
$$

\paperheading{Discovery}

Let us examine some function $f(x)$ which is periodic over an interval $\tau$,
meaning that
$$
f(t + \tau) = (t)
$$
We might hypothesize that this function can be approximated as a weighted sum
of $N$ other period functions $g_n(t)$:
$$
f(t) \approx \sum_{n = 0}^{N} a_n g_n(t)
$$
But what functions should we use for~$g_n(t)$? And how should we determine the
weights~$a_n$? The trigonometric functions might be a good place to start. The
most general form for a sine wave is
$$
A\cos(\omega t - \phi)
$$
where $\omega$ is some angular frequency, and $\phi$ is some phase. We can
also write this as
$$
\alpha \cos(\omega t) + \beta \sin(\omega t)
$$
where
$$
\eqalign{
A &= \sqrt{\alpha^2 + \beta^2} \cr
\phi &= \arctan(\?\beta/\alpha\?) \cr
}
$$
If we want this sine wave to have the period $\tau$ then we should make sure
that $\omega\tau$ is a multiple of $2\pi$:
$$
\omega \tau \bmod 2\pi = 0
$$
or in other words that
$$
\q{\omega \tau}{2\pi} \bmod 1 = 0
$$
or in other words that $\omega \tau/2\pi$ evaluates to some integer $n$
$$
\q{\omega \tau}{2\pi} = n
$$
Solving this for $\omega$ gives
$$
\omega = \q{2\pi n}{\tau}
$$

Now, from this result for $\omega$ it is obvious that there are infinitely
many sine waves with period $\tau$, because $n$ can take on any positive
value. Previously we had said that we were going to take $N$ function $g_n(t)$
to make an approximation for $f(t)$. Yet now we have infinitely many choices
for $g_n(t)$. Perhaps if we use all of them with the proper weights we will
find an exact representation:
$$
f(t) = \sum_{n = 0}^{\infty} a_nA_n\cos(\omega_n t - \phi_n)
$$
At this point we should notice that encoding the weight as $a_nA_n$ is
redundant. We might as well just encode the whole weight in $A_n$, and instead
write
$$
f(t) = \sum_{n = 0}^{\infty} A_n\cos(\omega_n t - \phi_n)
$$
or alternatively
$$
f(t) =
\sum_{n = 0}^{\infty} 
\alpha_n \cos(\omega_n t) + \beta_n \sin(\omega_n t)
$$

Keep in mind that the above equality is still just a speculation. The truth of
the above will depend on our ability to find values for $A_n$ and $\phi_n$
such that that the infinite sum converges to $f(x)$. It would do just as well
to find values for $\alpha_n$ and $\beta_n$, because we know how they can
later be transformed into $A_n$ and $\phi_n$.

It turns out that finding $\alpha_n$ and $\beta_n$ is easier than finding
$A_n$ and $\phi_n$. The method we are about to present might seem like a leap
of genius to those readers who have not studied the mathematics of
time-averages. Yet it will not come as surprise to those readers who are
familiar with time-averages, and who know that many time-averages can be made
to zero out. In order to make conditions conducive to this zeroing let us
multiply the whole relation by $\cos(\omega_m t)$, where $m$ is some arbitrary
integer:
$$
f(t)\cos(\omega_m t) =
\sum_{n = 0}^{\infty} 
\alpha_n \cos(\omega_n t)\cos(\omega_m t)
+ \beta_n \sin(\omega_n t)\cos(\omega_m t)
$$
We know that taking a time average of this equation over the period
$\tau$ will make almost every term drop to 0, save for the term $\alpha_n
\cos(\omega_n t)\cos(\omega_m t)$ where $m = n$, which will become
$\cos^2(\omega_n t)$ and will time average to $\alpha_n * \tau / 2$:
$$
\int_{-\tau/2}^{\tau/2}
f(t)\cos(\omega_n t) \, dt
=
\alpha_n \q{\tau}{2}
$$
And so we have
$$
\alpha_n 
=
\q{2}{\tau}
\int_{-\tau/2}^{\tau/2}
f(t)*\cos\(\q{2\pi}{\tau}n * t\) \, dt
$$
Doing the same analysis with $\sin(\omega_m t)$ then shows that
$$
\beta_n 
=
\q{2}{\tau}
\int_{-\tau/2}^{\tau/2}
f(t)*\sin\(\q{2\pi}{\tau}n * t\) \, dt
$$

\bye
