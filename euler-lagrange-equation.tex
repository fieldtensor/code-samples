\input ../paper.tex

\papertitle{The Euler-Lagrange Equation}{}

\paperheading{Summary}

\noindent
The Euler-Lagrange equation is
$$
\q{\p f}{\p y} = \q{d }{dt}\q{\p f}{\p \dot y}
$$
Any function $y$ which satisfies this equation for a given function $f$ will
cause the following integral $S$ to be locally minimum, maximum, or at least
stable relative to other functions in the neighborhood of $y$
$$
S = \int_{t_1}^{t_2} f(y, \dot y, t) \, dt
$$

\paperheading{Discovery}

\noindent
Let us examine the following integral
$$
S = \int_{t_1}^{t_2} f(y, \dot y, t) \, dt
$$
Consider the set $Y$ of all functions $y$ which have the property that $y(t_1)
= Q$ and $y(t_2) = P$. We seek to find a specific subset of functions within
$Y$ which put $S$ in a minimum position, maximum position, or at least in a
stable position with respect to other functions in their immediate vicinity.
If we perturb a function in this subset and then make use of the integral $S$
to examine the perturbed version we should find that there has been no
significant change in $S$ relative to the size of the perturbation.

Let us now carry out this perturbation analysis. We will say that the shape of
the perturbation is some function $\eta(t)$, and we will denote the
differential magnitude of the perturbation by the symbol $\varepsilon$. The
combined shape and magnitude are thus written as $\varepsilon\eta(t)$. We will
also specify that the perturbation is such that $\eta(t_1) = 0$ and $\eta(t_2)
= 0$, so that even once the perturbation is applied the resulting function
still goes from $Q$ to $P$, which is required for it to be in the set $Y$. The
integral $S$ over the perturbed function will be
$$
S =
\int_{t_1}^{t_2}
f(y + \varepsilon\eta, \dot y + \varepsilon\dot\eta, t) \, dt
$$
In order to shorten our notation we will call the perturbed path
$\xi(\varepsilon, t)$, and we will set $u(\varepsilon, t) = \[\xi(\varepsilon,
t), \dot\xi(\varepsilon, t), t\]$. We can now write the integral over the
perturbed path as
$$
S =
\int_{t_1}^{t_2} f(u(\varepsilon, t)) \, dt
$$
We can easily calculate the exact increase in the integral's value relative to
the magnitude of the perturbation.
$$
\eqalign{
\q{dS}{d\varepsilon}
&=
\int_{t_1}^{t_2}
\q{d}{d\varepsilon}
\mathinner{\Bigl[f(u(\varepsilon, t))\Bigr]}
\, dt \cr
&=
\int_{t_1}^{t_2}
\eta\q{\p f}{\p y}\Bigl(u(\varepsilon, t)\Bigr)
+ \dot\eta\q{\p f}{\p \dot y}\Bigl(u(\varepsilon, t)\Bigr)
\, dt \cr
&=
\int_{t_1}^{t_2} \eta\q{\p f}{\p y}\Bigl(u(\varepsilon, t)\Bigr) \, dt +
\left[
\eta\q{\p f}{\p \dot y}\Bigl(u(\varepsilon, t)\Bigr) -
\int \eta\q{d}{dt}\q{\p f}{\p \dot y}\Bigl(u(\varepsilon, t)\Bigr) \, dt
\right]_{t_1}^{t_2} \cr
&=
\int_{t_1}^{t_2} \eta\q{\p f}{\p y}\Bigl(u(\varepsilon, t)\Bigr) \, dt +
\left[
\eta\q{\p f}{\p \dot y}\Bigl(u(\varepsilon, t)\Bigr)
\right]_{t_1}^{t_2} -
\int_{t_1}^{t_2} \eta\q{d}{dt}\q{\p f}{\p \dot y}\Bigl(u(\varepsilon, t)\Bigr) \, dt \cr
&=
\int_{t_1}^{t_2}
\eta\[\q{\p f}{\p y}\Bigl(u(\varepsilon, t)\Bigr) - \q{d}{dt}\q{\p f}{\p \dot y}\Bigl(u(\varepsilon, t)\Bigr)\] dt
}
$$
We wish to examine the rate of change of $S$ with respect to $\varepsilon$ as
the perturbation is only just starting to be applied, and so we will set
$\varepsilon=0$. This will result in $u(\varepsilon, t)$ reducing back to the
ideal path $$u(0, t) = \[\xi(0, t), \dot\xi(0, t), t\] = \Bigl[y(t), \dot
y(t), t\Bigr]$$ Our derivative is now
$$
\q{dS}{d\varepsilon}\Bigl(0\Bigr)
=
\int_{t_1}^{t_2}
\eta\[\q{\p f}{\p y} - \q{d}{dt}\q{\p f}{\p \dot y}\] dt
=
0
$$
We have set this all to 0 because the integral is now being evaluated over one
of the ideal paths $y$ from the set $Y$, which by construction is a path that
causes $S$ to be stabilized. We now notice that the magnitude $\varepsilon$ of
the perturbation has been eliminated from our equation, but the term $\eta$
denoting the perturbation's shape still remains. This shape might be anything,
and everything. It certainly need not be 0 for all $t$, and it certainly need
not be well behaved enough to to render the integral always equal to 0. The
only way to guarantee that the integral comes out to 0 for all $\eta$ is to
ensure that
$$
\q{\p f}{\p y} - \q{d}{dt}\q{\p f}{\p \dot y} = 0
$$
or in other words that
$$
\q{\p f}{\p y} = \q{d}{dt}\q{\p f}{\p \dot y}
$$
This is called the Euler-Lagrange equation. Any function $y$ which satisfies
this equation for a given function $f$ will cause the integral $S$ to be
locally minimum, maximum, or at least stable.

\bye
