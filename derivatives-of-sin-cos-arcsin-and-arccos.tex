\input ../paper.tex

\papertitle{The Derivatives of sin, cos, arcsin, and arccos}

\paperheading{Summary}

\noindent
The derivatives of $\sin$ and $\cos$ are
$$
\eqalign{
\sin' &= \cos \cr
\cos' &= -\sin \cr
}
$$
The derivatives of $\arcsin$ and $\arccos$ are
$$
\eqalign{
\arcsin'x &= \q{1}{\sqrt{1 - x^2}} \cr
\noalign{\vskip6pt}
\arccos'(x) &= -\q{1}{\sqrt{1 - x^2}} \cr
}
$$

\paperheading{Derivation}

Consider the function that draws the upper half of the unit circle:
$$
f(x) = \sqrt{1 - x^2}
$$
To every $x$ we can assign some angle $\theta = \arccos(x)$. To that $\theta$
we can assign some \hbox{pie-shaped} slice of the unit circle, and this slice
will have area $A = \theta/2$. It also possible to define a function $A(x)$
which gives the area of that very same slice, but only in terms of~$x$:
$$
\eqalign{
A(x) &= \q12xf(x) + \int_x^1 f(x) \; dx
\cr
&= \q12x\sqrt{1 - x^2} + \int_x^1 \sqrt{1 - x^2} \; dx
\cr
}
$$
Let us explore this function $A(x)$ by finding its derivative
$$
\eqalign{
A'(x)
&= \q12\[\sqrt{1 - x^2} + x*\q12\(1 - x^2\)^{-1/2}*-2x\] - \sqrt{1 - x^2}
\cr
&= \q12\[\sqrt{1 - x^2} - \q{x^2}{\sqrt{1-x^2}} - 2\sqrt{1 - x^2}\] 
\cr
&= -\q12\[\sqrt{1 - x^2} + \q{x^2}{\sqrt{1-x^2}}\]
\cr
&= -\q12\[\q{1 - x^2 + x^2}{\sqrt{1 - x^2}}\]
\cr
&= -\q12\q{1}{\sqrt{1 - x^2}}
}
$$
Now, consider the expression
$$
A(\cos\theta)
$$
On the one hand we can evaluate this as
$$
A(\cos(\theta))
= \q12*\cos\theta*f(\cos\theta) + \int_{\cos\theta}^1 f(\cos\theta) \; dx
$$
Yet on the other hand when we look at $A(\cos\theta)$ we see that an angle is
going in and an area is coming out, and so we know immediately that
$$
A(\cos\theta) = \theta/2
$$
We thus have
$$
\q12\theta
= \q12*\cos\theta*f(\cos\theta) + \int_{\cos\theta}^1 f(x) \; dx
$$
Differentiating this will yield a novel result, and most of the work has
already been done:
$$
\eqalign{
\q12 &= A'(\cos\theta)*\cos'\theta
\cr
\q12 &= -\q12\q{1}{\sqrt{1 - \cos^2\theta}}\cos'\theta
\cr
1 &= -\q{1}{\sqrt{1 - \cos^2\theta}}\cos'\theta
\cr
\cos'\theta &= -\sin\theta
}
$$
This completes the proof. Finding $\sin'$ is then easy:
$$
\eqalign{
1 &= \sin^2 + \cos^2 \cr
0 &= 2\sin\sin' + 2\cos\cos' \cr
-2\sin\sin' &=  2\cos\cos' \cr
-\sin\sin' &=  - \cos\sin' \cr
\sin' &= \cos \cr
}
$$

\paperheading{The derivative of arccos and arcsin}

Notice that in the previous section we had
$$
A(\cos\theta) = \q12\theta
$$
If we simply set $\theta = \arccos(x)$ then we have
$$
A(x) = \q12\arccos(x)
$$
We can then differentiate this using the tools we have already developed:
$$
\eqalign{
A'(x) &= \q12\arccos'(x)
\cr
-\q12\q{1}{\sqrt{1 - x^2}} &= \q12\arccos'(x)
}
$$
And so we see that
$$
\arccos'(x) = -\q{1}{\sqrt{1 - x^2}}
$$
The derivative $\arcsin'$ is then easy to find:
$$
\eqalign{
\sin(\arcsin x) &= x \cr
\sin'(\arcsin x) * \arcsin'x &= 1 \cr
\arcsin'x &= \q{1}{\sin'(\arcsin x)} \cr
\arcsin'x &= \q{1}{\cos(\arcsin x)} \cr
\arcsin'x &= \q{1}{\sqrt{1 - \sin^2(\arcsin x)}} \cr
\arcsin'x &= \q{1}{\sqrt{1 - x^2}} \cr
}
$$


\bye
